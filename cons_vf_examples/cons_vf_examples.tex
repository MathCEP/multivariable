\documentclass{ximera}  

\title{Conservative Vector Fields and Topology Extra Examples}
\author{Melissa Lynn}  

\begin{document}  
\begin{abstract}  
%These are extra examples for the sections on Conservative Vector Fields and Topology.
\end{abstract}  
\maketitle


\begin{example}
Determine whether or not the vector field $\mathbf{F}(x,y) = (-\sin(x)\sin(y)+2xye^{x^2y},\cos(x)\cos(y)+x^2e^{x^2y})$ is conservative. If it is conservative, find a potential function.
\begin{explanation}
Before we attempt to find a potential function, let's check if the derivative matrix $D\mathbf{F}$ is symmetric. Note that we only need to compare the $(0,1)$ and $(1,0)$ entries in order to check if this matrix is symmetric.
\[
D\mathbf{F}(x,y) = \left(\begin{array}{cc}
*&-\sin(x)\cos(y)+2xe^{x^2y}+2x^3ye^{x^2y}\\
-\sin(x)\cos(y)+2xe^{x^2y}+2x^3ye^{x^2y}&*\end{array}\right)
\]
We see that the derivative matrix is symmetric, so we suspect that this vector field is conservative. In order to verify this, we'll find a potential function.
Suppose $f(x,y)$ is a potential function for $\mathbf{F}$. Then we must have
\[
\nabla f(x,y)=\mathbf{F}(x,y)=(-\sin(x)\sin(y)+2xye^{x^2y},\cos(x)\cos(y)+x^2e^{x^2y}+y^2).
\]
This means that $\frac{\partial f}{\partial x} = -\sin(x)\sin(y)+2xye^{x^2y}$ and $\frac{\partial f}{\partial y} = \cos(x)\cos(y)+x^2e^{x^2y}+y^2$. We'll start by considering $\frac{\partial f}{\partial x}$. Integrating with respect to $x$, we have
\begin{align*}
f(x,y) &= \int -\sin(x)\sin(y)+2xye^{x^2y}\,dx\\
&= \cos(x)\sin(y)+e^{x^2y}+g(y)\\
\end{align*}
where $g(y)$ is a function of $y$. So, we know that $f$ has the form $f(x,y)=\cos(x)\sin(y)+e^{x^2y}+g(y)$. We differentiate this with respection to $y$ in order to compare with the $y$-component of the vector field $\mathbf{F}$.
\begin{align*}
\frac{\partial f}{\partial y} &= \frac{\partial}{\partial y}\left(\cos(x)\sin(y)+e^{x^2y}+g(y)\right)\\
&= \cos(x)\cos(y) + x^2e^{x^2y}+g'(y)
\end{align*}
Comparing this with $\frac{\partial f}{\partial y} = \cos(x)\cos(y)+x^2e^{x^2y}+y^2$, we conclude that $g'(y)=y^2$, which we integrate with respect to $y$ in order to find $g$.
\begin{align*}
g(y) &= \int y^2\,dx\\
&=\frac{1}{3}y^3+C
\end{align*}
Thus, we have that any potential function must have the form $f(x,y) = \cos(x)\sin(y)+e^{x^2y}+\frac{1}{3}y^3+C$. Taking $C=0$, we obtain a specific potential function $f(x,y) = \cos(x)\sin(y)+e^{x^2y}+\frac{1}{3}y^3$.

Since $f$ is a potential function for $\mathbf{F}$, we conclude that $\mathbf{F}$ is conservative.
\end{explanation}
\end{example}


\end{document}