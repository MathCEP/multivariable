\documentclass{ximera}  

\title{Connections of Curl with Older Material}  

\begin{document}  
\begin{abstract}  
%Connections
\end{abstract}  
\maketitle 

We've defined the curl of a two or three dimensional vector field, and we found that this gives a measure of the local rotation of a vector field.

In this section, we discuss connections of the curl to previous topics from the course. In particular, we find the curl of a conservative vector field, and we restate Green's Theorem in terms of curl.

\section{Curl of a Conservative Vector Field}

In this section, we prove that the curl of a conservative vector field will always be zero. Thus, conservative vector fields are irrotational.

\begin{theorem}
Suppose $\mathbf{F}$ is a $C'$ conservative vector field in $\mathbb{R}^3$, so there is a function $f:\mathbb{R}^3\rightarrow\mathbb{R}$ such that $\mathbf{F}=\nabla f$ Then $\nabla\times\mathbf{F} = \mathbf{0}$.
\end{theorem}

\begin{proof}
Suppose $\mathbf{F}(x,y,z)=(M,N,P)$ is a $C^1$ conservative vector field, with $\mathbf{F}=\nabla f$. Then we must have
\[
\left(\dfrac{\partial f}{\partial x}, \dfrac{\partial f}{\partial y}, \dfrac{\partial f}{\partial z}\right) = \answer{(M,N,P)}.
\]
Computing the curl of $\mathbf{F}$, we have
\begin{align*}
\nabla\times\mathbf{F} &= \left(\dfrac{\partial P}{\partial y}-\dfrac{\partial N}{\partial z}, -\left(\dfrac{\partial P}{\partial x}-\dfrac{\partial M}{\partial z}\right), \dfrac{\partial N}{\partial x} - \dfrac{\partial M}{\partial y}\right)\\
&= \left(\dfrac{\partial}{\partial y}\dfrac{\partial f}{\partial z}-\dfrac{\partial}{\partial z}\dfrac{\partial f}{\partial y}, -\left(\dfrac{\partial}{\partial x}\dfrac{\partial f}{\partial z}-\dfrac{\partial}{\partial z}\dfrac{\partial f}{\partial x}\right), \dfrac{\partial}{\partial x}\dfrac{\partial f}{\partial y} - \dfrac{\partial}{\partial y}\dfrac{\partial f}{\partial x}\right),
\end{align*}
substituting in for the components $M$, $N$, and $P$.

Now, we will use Clairaut's Theorem to simplify this vector. Since $\mathbf{F}=(M,N,P)$ is a $C^1$ vector field, the partial derivatives of its components ($\dfrac{\partial M}{\partial y}$, $\dfrac{\partial M}{\partial z}$, etc.) exist and are continuous. This means that all second-order partial derivatives of $f$ exist and are continuous. Then, by Clairaut's Theorem, the order of differentiation for the second-order mixed partials doesn't matter. In particular, we have
\begin{align*}
\dfrac{\partial^2 f}{\partial y\partial x} &= \dfrac{\partial^2 f}{\partial x\partial y}\\
\dfrac{\partial^2 f}{\partial z\partial x} &= \dfrac{\partial^2 f}{\partial x\partial z}\\
\dfrac{\partial^2 f}{\partial z\partial y} &= \dfrac{\partial^2 f}{\partial y\partial z}.
\end{align*}
Using this fact in our computation of the curl, we now have
\begin{align*}
\nabla\times\mathbf{F} &= \left(\dfrac{\partial^2 f}{\partial y\partial z}-\dfrac{\partial^2 f}{\partial z\partial y}, -\left(\dfrac{\partial^2 f}{\partial x\partial z}-\dfrac{\partial^2 f}{\partial z\partial x}\right), \dfrac{\partial^2 f}{\partial x\partial y} - \dfrac{\partial^2 f}{\partial y\partial x}\right) \\
&= \answer{(0,0,0)}
\end{align*}
Thus, we have shown that the curl of a conservative vector field is zero.
\end{proof}

So, conservative vector fields are irrotational. A reasonable follow-up question would be: if the curl of a vector field is zero, is the vector field necessarily conservative? We'll leave this as an open question for the reader, with the suggestion that you think about how you can use past results, and what hypotheses are necessary for this converse to be true.


\section{Curl and Green's Theorem}

In this section, we see that we've actually already seen the curl of a vector field. It turns out that the curl showed up in Green's Theorem, we just didn't know that it was the curl yet.

Recall the statement of Green's Theorem:

\begin{theorem}
Let $D$ be a closed and bounded region in $\mathbb{R}^2$, whose boundary $\partial D$ consists of finitely many simple, closed, piecewise $C^1$ curves. Orient the boundary $\partial D$ so that $D$ is on the left as one travels along $\partial D$.

Let $\mathbf{F}(x,y)=(M,N)$ be a $C^1$ vector field defined on $D$. Then,
\[
\oint_C\mathbf{F}\cdot d\mathbf{s} = \iint_D\left(\dfrac{\partial N}{\partial x}-\dfrac{\partial M}{\partial y}\right)\,dxdy.
\]
\end{theorem}

The integrand of the double integral, $\dfrac{\partial N}{\partial x}-\dfrac{\partial M}{\partial y}$, should now look familiar. This mysterious quantity is actually the two-dimensional curl of the vector field $\mathbf{F}$!

Using this realization, we can now restate Green's Theorem in terms of the curl of $\mathbf{F}$.

\begin{theorem}
Let $D$ be a closed and bounded region in $\mathbb{R}^2$, whose boundary $\partial D$ consists of finitely many simple, closed, piecewise $C^1$ curves. Orient the boundary $\partial D$ so that $D$ is on the left as one travels along $\partial D$.

Let $\mathbf{F}(x,y)=(M,N)$ be a $C^1$ vector field defined on $D$. Then,
\[
\oint_C\mathbf{F}\cdot d\mathbf{s} = \iint_D\nabla\times\mathbf{F}\,dxdy.
\]
\end{theorem}

Now, let's think a bit more about what Green's Theorem is saying here.

The vector line integral, $\oint_C\mathbf{F}\cdot d\mathbf{s}$, computes the global circulation of the vector field around the boundary of the region.

The double integral $\iint_D\nabla\times\mathbf{F}\,dxdy$ is computed by integrating curl over the region $D$. We can think of this as ``adding up'' the local rotation of the vector field.

Thus, we can think of Green's Theorem as saying that the global circulation of the vector field around the boundary is equal to the total local rotation across the region. If you think about it, this does make some sense!

\section{Summary}

In this activity, we connected the curl of a vector field to concepts we've covered previously. In particular, we showed that the curl of a $C^1$ conservative vector field is zero, and we restated Green's Theorem in terms of curl.

\end{document}