\documentclass{ximera}
\title{Graphing Functions}
\begin{document}
\begin{abstract}
\end{abstract}
\maketitle

In this activity, we give the formal definition of the graph of a function $f:\mathbb{R}^2\rightarrow\mathbb{R}$. We discuss strategies for figuring out the shapes of the graph of such functions, using contour curves, level curves, and sections.

\section{Definition of the Graph of a Function}

You might already have an intuitive idea of what the graph of a function $f:X\subset\mathbb{R}^2\rightarrow\mathbb{R}$ should be, but perhaps don't know the formal definition, or how to figure out what the graph of an arbitrary function looks like. We'll begin with the definition of the graph, before discussing how to actually produce graphs.

\begin{definition}
Let $f:X\subset\mathbb{R}^2\rightarrow\mathbb{R}$ be a function. The \emph{graph} of $f$ is the set of points
\[
\textrm{Graph }f = \{(\vec{x},f(\vec{x})\,:\,\vec{x}\in X\}
\]
in $\mathbb{R}^3$.

We typically visualize a point in the graph as lying over the point $\vec{x}$ in the plane at a height $f(\vec{x})$.
\end{definition}

Note that this is similar to the graph of a function from a subset of $\mathbb{R}$ to $\mathbb{R}$. The graph of a function $X\subset\mathbb{R}^n\rightarrow\mathbb{R}$ can be defined similarly, but this tends to be less useful once $n\geq 3$, since it's hard to visualize four or more dimensions!

\section{Strategies for Graphing}

It can be much trickier to sketch the graph of a function $f:X\subset\mathbb{R}^2\rightarrow\mathbb{R}$ than it was to sketch the graphs of functions $\mathbb{R}\rightarrow\mathbb{R}$. One common strategy that people will initially try is plotting individual points to try to get a sense of the graph. However, for graphs in $\mathbb{R}^3$, you would need a lot of points to get a representative sample of the plane. For this reason, \emph{plotting points alone is not an effective strategy}. However, plotting a single point here or there can be helpful.

We've now told you what doesn't work for graphing functions in $\mathbb{R}^3$, so now we should probably tell you what does work. The essential idea of all of these strategies is that we know you're pretty comfortable graphing in $\mathbb{R}^2$, so we're going to take advantage of that experience.

We'll begin with contour curves, which are obtained by setting the $z$-coordinate to be constant. Think of this as taking horizontal slices of the graph.

\begin{definition}
Let $f:X\subset\mathbb{R}^2\rightarrow\mathbb{R}$ be a function. The \emph{contour curve} of the function $f$ at height $C$ is the set of points in $\mathbb{R}^3$ obtained by taking the intersection of the graph of $f$ with the plane $z=C$.
\end{definition} 

PICTURE/VIDEO EXAMPLE

We can also consider the level curves of a function, which are closely related to contour curves.

\begin{definition}
Let $f:X\subset\mathbb{R}^2\rightarrow\mathbb{R}$ be a function. The \emph{level curve} of the function $f$at height $C$ is the set of points in $\mathbb{R}^2$ satisfying $C = f(x,y)$.
\end{definition}

After reading this definition, you're probably thinking ``hey, aren't contour curves and level curves the exact same thing?'' They're certainly closely related. The key difference is that level curves exist in the plane, $\mathbb{R}^2$, while contour curves exist in three-space, $\mathbb{R}^3$. Since they're in the plane, level curves are usually easier to draw. However, contour curves are more useful for figuring out the shape of a graph. For these reasons, it can be useful to go back and forth between level curves and contour curves.

PICTURE/VIDEO EXAMPLE

We can think of contour curves as taking slices of the graph where $z$ is constant. It can also be useful to take slices of the graph where $x$ or $y$ is constant. We call these slices sections of the graph.

\begin{definition}
Let $f:X\subset\mathbb{R}^2\rightarrow\mathbb{R}$ be a function, and let $C$ be a constant.

The \emph{section} of the graph of $f$ by $x=C$ is the set of points 
\[
\{(C,y,z)\in\mathbb{R}^3\,:\,z = f(C,y)\}
\]

The \emph{section} of the graph of $f$ by $y=C$ is the set of points 
\[
\{(x,C,z)\in\mathbb{R}^3\,:\,z = f(x,C)\}
\]
\end{definition}

Note that, like contour curves, sections exist in $\mathbb{R}^3$.

PICTURE/VIDEO EXAMPLE

EXTRA EXAMPLE

\section{Level Surfaces}

So far, we have focused on graphing functions from subsets of $\mathbb{R}^2$ to $\mathbb{R}$, so the graphs are in $\mathbb{R}^3$.

We now turn our attention to the graphs of functions from subsets of $\mathbb{R}^3$ to $\mathbb{R}$. Note that the graph of such a function will exist in $\mathbb{R}^4$. Since the world we live in only has three physical dimensions, it can be very difficult to visualize a four dimensional object! Fortunately, there are various tricks that can be used to get some sense of what a four dimensional object looks like. We cover one of them here.

When we had a function $f:D\subset\mathbb{R}^2\rightarrow\mathbb{R}$, we could get a sense of the graph by looking at its level curves, which were curves in the same plane.

For a function $f:D\subset\mathbb{R}^3\rightarrow\mathbb{R}$, we can adopt a similar approach. We can once again consider the level sets, which are obtained by taking the output to be some constant:
\[
f(x,y,z) = C.
\]
In this case, the level sets will be level surfaces, which live in $\mathbb{R}^3$. By graphing several level surfaces, we can see what a slice of the graph of $f$ looks like at various heights, giving us some sense of how the overall graph behaves. Of course, because this graph exists in four dimensions, we still probably won't be able to visualize this perfectly.

To see how this can help us visualize the four-dimensional graph of a function $f:\mathbb{R}^3\rightarrow\mathbb{R}$, we give an example.

\begin{example}
Consider the function
\[
f(x,y,z) = \sqrt{x^2 + y^2 + z^2}.
\]
Find the level surfaces at heights $-1$, $0$, $1$, $2$, and $3$. Use these level surfaces to describe the graph of $f$.

We'll begin with the level surface at height $-1$. This is the set of points $(x,y,z)$ in $\mathbb{R}^3$ such that
\[
-1 = \sqrt{x^2+y^2+z^2}.
\]
There are no points that satisfy this equation, so the level surface is empty.

Now we'll consider the level surface at height $0$. This is the set of points $(x,y,z)$ in $\mathbb{R}^3$ such that
\[
0 = \sqrt{x^2+y^2+z^2}.
\]
The only point which satisfies this equation is the origin, so the level ``surface'' is the single point $(0,0,0)$.

Let's look at the level surface at height $1$. This is the set of points $(x,y,z)$ in $\mathbb{R}^3$ such that
\[
1 = \sqrt{x^2+y^2+z^2}.
\]
Squaring both sides, we can rewriting this as 
\[
1 = x^2+y^2+z^2.
\]
The graph of this equation is the sphere of radius $1$ centered at the origin, which is our level surface.

Let's look at the level surface at height $2$. This is the set of points $(x,y,z)$ in $\mathbb{R}^3$ such that
\[
2 = \sqrt{x^2+y^2+z^2}.
\]
Squaring both sides, we can rewriting this as 
\[
4 = x^2+y^2+z^2.
\]
The graph of this equation is the sphere of radius $2$ centered at the origin, which is our level surface.

Let's look at the level surface at height $3$. This is the set of points $(x,y,z)$ in $\mathbb{R}^3$ such that
\[
3 = \sqrt{x^2+y^2+z^2}.
\]
Squaring both sides, we can rewriting this as 
\[
9 = x^2+y^2+z^2.
\]
The graph of this equation is the sphere of radius $3$ centered at the origin, which is our level surface.

We graph our level surfaces below.

PICTURE

We can see that the level surfaces are spheres whose radii increase linearly with the height. So, we can describe the graph of $f$ as some sort of four-dimensional cone.

\end{example}


\section{Conclusion}

In this activity, we gave the formal definition of the graph of a function $f:\mathbb{R}^2\rightarrow\mathbb{R}$. We discussed strategies for figuring out the shapes of the graph of such functions, using contour curves, level curves, and sections.

\end{document}