\documentclass{ximera}

\graphicspath{{./graphics/}}

\title{Continuity and Limits in General}
\begin{document}
\begin{abstract}
\end{abstract}
\maketitle

So far, we've seen how we can show that limits exist using a delta-epsilon proof, or by changing coordinates. In single variable calculus, we were often able to evaluate limits by direct substitution. For example, we could evaluate
\begin{align*}
\lim_{x\rightarrow 2} x^2+x\sin(\pi x) &= 2^2 + 2\sin(\pi \cdot 2)\\
&= \answer{4}. 
\end{align*}
We are able to do this because the function $f(x) = x^2+x\sin(\pi x)$ is continuous. Recall that a function $f:\mathbb{R}\rightarrow \mathbb{R}$ is \emph{continuous} at $x=a$ if
\[
\lim_{x\rightarrow a}f(x) = f(a).
\]
For continuous functions, we can evaluate limits by simply plugging in the value.

Once we define continuity for multivariable functions, and determine which functions are continuous, we can use similar methods to evaluate multivariable limits.

\section*{Continuity}

We define continuity similarly to how we did in single variable calculus.

\begin{definition}
A function $f:\mathbb{R}^n\rightarrow\mathbb{R}$ is \emph{continuous} at $\vec{x}=\vec{a}$ in $\mathbb{R}^n$ if
\[
\lim_{\vec{x}\rightarrow\vec{a}}f(\vec{x}) = f(\vec{a}).
\]
\end{definition}

Also similar to single variable calculus, virtually all of the common functions that we work with are continuous on their domains. That is, anywhere that they're defined, they are continuous.

\begin{theorem}
The following functions are continuous on their domains:
\begin{itemize}
\item polynomials
\item root functions
\item rational functions
\item trigonometric functions and inverse trigonometric functions
\item exponential functions and logarithmic functions
\end{itemize}
\end{theorem}

Furthermore, all of the ways that we'd like to combine continuous functions will result in another continuous function.

\begin{theorem}
If $f$ is continuous at $\vec{x}=\vec{a}$, and $k$ is a real number, then $kf$ is also continuous at $\vec{a}$.

If $f$ and $g$ are continuous at $\vec{x}=\vec{a}\in\mathbb{R}^n$, then $f+g$ is also continuous at $\vec{a}$.

If $f$ and $g$ are continuous at $\vec{x}=\vec{a}\in\mathbb{R}^n$, then $fg$ is also continuous at $\vec{a}$.

If $g$ is continuous at $\vec{x}=\vec{a}$, and $f$ is continuous at $g(\vec{a})$, then $f\circ g$ is continuous at $\vec{a}$.
\end{theorem}

\begin{example}
Which of the following functions is continuous at $(0,0)$? Select all that apply.
\begin{selectAll}
\choice[correct]{$f(x,y) = 3x^3+2xy^2+x+1$}
\choice[correct]{$g(x,y) = \sin(x)\cos(x)$}
\choice[correct]{$h(x,y) = \frac{x^2+y^2+1}{x+y+1}$}
\choice{$i(x,y) = \frac{x^2y}{x^2+y^2}$}
\choice[correct]{$j(x,y) = \tan(xy)$}
\choice[correct]{$k(x,y) = e^{\sin(x+y)}$}
\choice{$l(x,y) = \ln(x^2+y^2)$}
\choice[correct]{$m(x,y) = \frac{1}{\ln(x^2+y^2+2)}$}
\end{selectAll}
\end{example}

\begin{example}
Evaluate the following limits, or enter ``DNE'' if they do not exist.
\begin{align*}
\lim_{(x,y)\rightarrow(0,0)}3x^3+2xy^2+x+1& = \answer{1}\\
\lim_{(x,y)\rightarrow(0,0)} \sin(x)\cos(x) &= \answer{0}\\
\lim_{(x,y)\rightarrow(0,0)}\frac{x^2+y^2+1}{x+y+1} &= \answer{1}\\
\lim_{(x,y)\rightarrow(0,0)} \frac{x^2y}{x^2+y^2} &= \answer{0}\\
\lim_{(x,y)\rightarrow(0,0)}\tan(xy) &= \answer{0}\\
\lim_{(x,y)\rightarrow(0,0)}e^{\sin(x+y)} &= \answer{1}\\
\lim_{(x,y)\rightarrow(0,0)}\ln(x^2+y^2) &= \answer{DNE}\\
\lim_{(x,y)\rightarrow(0,0)}\frac{1}{\ln(x^2+y^2+2)} &= \answer{1/\ln(2)}
\end{align*}
Note that even if a function is discontinuous at a point, it's still possible that the limit exists. In this case, you'll need to use a change of coordinates or other method to evaluate the limit.
\end{example}

\section*{Limits in General}

So far, we've defined limits of scalar-valued functions, $\mathbb{R}^n\rightarrow\mathbb{R}$. We've seen how we can evaluate these limits, or show that they do not exist. However, we've yet to deal with more general multivariable functions, $\mathbb{R}^n\rightarrow\mathbb{R}^m$.

Fortunately, limits in the more general setting turn out to be an easy extension of the limits that we've already defined. That is, if we have a function $\vec{f}:\mathbb{R}^n\rightarrow\mathbb{R}^m$, we can write $\vec{f}$ in terms of its coordinate functions,
\[
\vec{f}(\vec{x}) = (f_1(\vec{x}), f_2(\vec{x}),...,f_m(\vec{x})).
\]
Then, we can use the limits of the coordinate functions to define a limit of $f$.

\begin{definition}
Suppose we have a function $\vec{f}:\mathbb{R}^n\rightarrow\mathbb{R}^m$ with $\vec{f}(\vec{x}) = (f_1(\vec{x}), f_2(\vec{x}),...,f_m(\vec{x}))$. Then we define
\[
\lim_{\vec{x}\rightarrow\vec{a}}\vec{f}(\vec{x}) = \left(\lim_{\vec{x}\rightarrow\vec{a}}f_1(\vec{x}), \lim_{\vec{x}\rightarrow\vec{a}}f_2(\vec{x}),...,\lim_{\vec{x}\rightarrow\vec{a}}f_m(\vec{x})\right),
\]
provided each of these limits exist.
\end{definition}

So, we can evaluate the limit of a function $\vec{f}:\mathbb{R}^n\rightarrow\mathbb{R}^m$ by taking the limit of the component functions. Because of this, the results and methods that we've used for limits of scalar functions carry naturally over to this more general setting.

\begin{example}
Evaluate the following limits, or enter ``DNE'' if they do not exist.
\begin{align*}
\lim_{(x,y,z)\rightarrow (1,2,3)}(x^2+y, z, xz) &= \answer{(3,3,3)}\\
\lim_{(x,y)\rightarrow(0,1)}\left(\sin(x+y), \frac{1}{\ln(y)}\right) &= \answer{DNE}\\
\lim_{(x,y)\rightarrow(0,0)}\left(\frac{x^3y}{x^2+y^2}, \frac{xy^3}{x^2+y^2}\right) &= \answer{(0,0)}
\end{align*}
\end{example}



\end{document}