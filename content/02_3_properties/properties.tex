\documentclass{ximera}

\graphicspath{{./graphics/}}

\title{Properties of Velocity and Speed}
\begin{document}
\begin{abstract}
\end{abstract}
\maketitle

In single variable calculus, we used the product rule to differentiate products of functions. For example, for $f(x) = x^2$ and $g(x) = \sin(x)$, we can find the derivative of $(fg)(x) = x^2\sin(x)$ as
\begin{align*}
(fg)'(x) &= f'(x)g(x)+f(x)g'(x)\\
&= 2x\sin(x)+x^2\cos(x).
\end{align*}
Although we can't take the product of two vectors in general, we do have the dot product and cross product, and we would like to understand how differentiation interacts with these products. Fortunately, they turn out to be very similar to the product rule from single variable calculus. 

\section*{Product rules}

\begin{proposition}
Consider paths $\vec{x}$ and $\vec{y}$ in $\mathbb{R}^n$. For $t$ such that $\vec{x}'(t)$ and $\vec{y}'(t)$ both exist, we have
\[
(\vec{x}\cdot\vec{y})'(t) = \vec{x}'(t)\cdot \vec{y}(t) + \vec{x}(t)\cdot \vec{y}'(t).
\]
If $n=3$, we also have
\[
(\vec{x}\times \vec{y})'(t) = \vec{x}'(t)\times \vec{y}(t) + \vec{x}(t)\times \vec{y}'(t).
\]
\end{proposition}

\begin{proof}
We prove this result for the dot product, and leave the proof for the cross product as an exercise.

Suppose $t$ is such that both $\vec{x}'(t)$ and $\vec{y}'(t)$ exist, and write $\vec{x}(t) = (x_1(t),...,x_n(t))$ and $\vec{y}(t) = (y_1(t),...,y_n(t))$. Then we have
\[
(\vec{x}\cdot\vec{y})(t) = x_1(t)y_1(t) + \cdots + x_n(t)y_n(t).
\]
Using the single variable product rule and regrouping, we have
\begin{align*}
(\vec{x}\cdot\vec{y})'(t) & = \frac{d}{dt}\left(x_1(t)y_1(t) + \cdots + x_n(t)y_n(t)\right),\\
& = x_1'(t) y_1(t) + x_1(t)y_1'(t) +\cdots+x_n'(t) y_n(t) + x_n(t)y_n'(t),\\
& = \left(x_1'(t) y_1(t) + \cdots + x_n'(t) y_n(t)\right) + \left(x_1(t) y_1'(t) + \cdots + x_n(t) y_n'(t)\right).
\end{align*}
Notice that the left summand is $\vec{x}'(t)\cdot\vec{y}(t)$ and the right summand is $\vec{x}(t)\cdot\vec{y}'(t)$. Thus, we arrive at our result,
\[
(\vec{x}\cdot\vec{y})'(t) = \vec{x}'(t)\cdot \vec{y}(t) + \vec{x}(t)\cdot \vec{y}'(t).
\]
\end{proof}


\section*{Paths on spheres}

Dot products can be used to tell us important information about the geometry of vectors. In particular, two vectors $\vec{v}$ and $\vec{w}$ in $\mathbb{R}^n$ are perpendicular if and only if $\vec{v}\cdot\vec{w}=\answer{0}$. Furthermore, the length of a vector $\vec{v}$ can be computed using the dot product, as $\|\vec{v}\| =\sqrt{\vec{v}\cdot\vec{v}}$.

We'll use these observations to determine the behavior of paths in a special case: when we have a curve which lies on a sphere.

Suppose $\vec{x}(t)$ lies on a sphere of radius $C$. In this case, for the position vector $\vec{x}(t)$, we have $\|\vec{x}(t)\|=C$. We can rewrite this in terms of dot products as $\vec{x}(t)\cdot\vec{x}(t)=C^2$. Differentiating this equation and using our above observations about dot products, it can be shown that $\vec{x}(t)$ and $\vec{x}'(t)$ are perpendicular.

Thus, for any parametrization of a curve which lies on a sphere, the velocity vector will always be perpendicular to the position vector.

\begin{example}
Let's verify this result for the path $\vec{x}(t) = (\cos(t^3+t)\sin(t), \sin(t^3+t)\sin(t), \cos(t)$ in $\mathbb{R}^3$, which lies on the unit sphere. Since this path lies on a sphere, we'd expect that
\begin{multipleChoice}
\choice{$\vec{x}(t)$ and $\vec{x}'(t)$ are parallel.}
\choice[correct]{$\vec{x}(t)$ and $\vec{x}'(t)$ are perpendicular.}
\choice{$\vec{x}'(t)$ is the zero vector.}
\choice{$\vec{x}'(t)$ also lies on the sphere.}
\end{multipleChoice}

First, we'll verify that this path lies on the sphere of radius $1$ by computing $\|\vec{x}(t)\|$. Repeatedly using the Pythagorean identity $\sin^2(x)+\cos^2(x)=1$, we have
\begin{align*}
\|\vec{x}(t)\| &= \sqrt{\cos^2(t^3+t)\sin^2(t) + \sin^2(t^3+t)\sin^2(t) + \cos^2(t)}\\
&= \sqrt{(\cos^2(t^3+t) + \sin^2(t^3+t))\sin^2(t) + \cos^2(t)}\\
&= \sqrt{1\cdot\sin^2(t) + \cos^2(t)}\\
&= \sqrt{1}\\
&= 1.
\end{align*}
Now, let's compute the velocity vector $\vec{x}'(t)$. Using the product rule and chain rule, we have
\[
\vec{x}'(t) = (-\sin(t^3+t)(3t^2+1)\sin(t) + \cos(t^3+t)\cos(t), \cos(t^3+t)(3t^2+1)\sin(t) + \sin(t^3+t)\cos(t), -\sin(t)).
\]
To check that $\vec{x}'(t)$ is perpendicular to $\vec{x}(t)$, we compute their dot product,
\begin{align*}
\vec{x}(t)\cdot \vec{x}'(t) =& \left(\cos(t^3+t)\sin(t)\right)\left(-\sin(t^3+t)(3t^2+1)\sin(t) + \cos(t^3+t)\cos(t)\right)\\
& + \left(\sin(t^3+t)\sin(t)\right)\left(\cos(t^3+t)(3t^2+1)\sin(t) + \sin(t^3+t)\cos(t)\right) + \left(\cos(t)\right)\left(-\sin(t)\right).
\end{align*}
Incredibly, this unwieldy expression simplifies very nicely.
\begin{align*}
\vec{x}(t)\cdot \vec{x}'(t) =& -(3t^2+1)\cos(t^3+t)\sin(t^3+t)\cos(t)\sin(t) + \cos^2(t^3+t)\cos(t)\sin(t)\\
&+ (3t^2+1)\cos(t^3+t)\sin(t^3+t)\cos(t)\sin(t) + \sin^2(t^3+t)\cos(t)\sin(t)-\cos(t)\sin(t)\\
=& \cos^2(t^3+t)\cos(t)\sin(t) +  \sin^2(t^3+t)\cos(t)\sin(t)-\cos(t)\sin(t)\\
=& \cos(t)\sin(t)-\cos(t)\sin(t)\\
=& 0.
\end{align*}
Thus, we have verifies that $\vec{x}(t)$ and $\vec{x}'(t)$ are perpendicular.

The path is graphed below for $-2\leq t\leq 2$, and we show how the velocity vector changes as $t$ increases. The position vector is shown in blue, and the velocity vector in black. Note that they are always perpendicular.

\youtube{BZ78A01v5yk}

\end{example}

\textit{Images were generated using \href{https://www.monroecc.edu/faculty/paulseeburger/calcnsf/CalcPlot3D/}{CalcPlot3D}.}

\end{document}