\documentclass{ximera}

\graphicspath{{./graphics/}}

\title{The Gradient}
\begin{document}
\begin{abstract}
\end{abstract}
\maketitle

We've given a formal definition for differentiability of a function $f:\mathbb{R}^2\rightarrow\mathbb{R}$,

\begin{definition}
Consider the function $f:\mathbb{R}^2\rightarrow\mathbb{R}$, and suppose that the partial derivatives $f_x$ and $f_y$ are defined at the point $(x,y)=(a,b)$. Define the linear function
\[
h(x,y) = f(a,b) + f_x(a,b)(x-a)+f_y(a,b)(y-b).
\]
We say that $f$ is \emph{differentiable} at $(x,y) = (a,b)$ if
\[
\lim_{(x,y)\rightarrow (a,b)}\frac{f(x,y) - h(x,y)}{\|(x,y)-(a,b)\|} = 0.
\]
If either of the partial derivatives $f_x(a,b)$ and $f_y(a,b)$ do not exist, or the above limit does not exist or is not $0$, then $f$ is not differentiable at $(a,b)$.
\end{definition}

The idea behind this definition is that $h(x,y)$ will be a ``good'' linear approximation to $f(x,y)$ near $(a,b)$ if $f$ is differentiable at $(a,b)$.

We would now like to define differentiability for scalar-valued functions of more than two variables, so functions $\mathbb{R}^n\rightarrow \mathbb{R}$. This definition will closely resemble our definition above, which handles the case $n=2$. For example, in the case $n=3$, we will use the linear function
\[
h(x,y,z) = f(a,b,c) + f_x(a,b,c)(x-a) + f_y(a,b,c)(y-b)+f_z(a,b,c)(z-c).
\]
For larger $n$, we'll define a similar function $h$, but this notation will quickly become unwieldy! In order to simplify notation, we'll now introduce a new object to organize our partial derivatives: the gradient of a scalar-valued function.

\section*{The gradient}

In order to organize our information about partial derivatives, and streamline our definition of differentiability for functions $\mathbb{R}^n\rightarrow\mathbb{R}$, we now define the gradient of a scalar-valued function.

\begin{definition}
Consider a function $f:\mathbb{R}^n\rightarrow\mathbb{R}$. The \emph{gradient} of $f$ is the function $\nabla f:\mathbb{R}^n\rightarrow\mathbb{R}^n$, defined by
\[
\nabla f = \left(\frac{\partial f}{\partial x_1},\frac{\partial f}{\partial x_2},...,\frac{\partial f}{\partial x_n}\right).
\]
\end{definition}

The gradient vector will be a useful computation tool in general, not only for defining differentiability.

\begin{example}
For $f(x,y,z) = x^2+ye^z$, we can compute the partial derivatives
\begin{align*}
f_x(x,y,z) &= 2x,\\
f_y(x,y,z) &= e^z,\\
f_z(x,y,z) &= ye^z. 
\end{align*} 
Then the gradient of $f$ is
\[
\nabla f = (2x, e^z ye^z).
\]
\end{example}

\begin{problem}
Find the gradient of each function.

$f(x,y,z) = \sin(xyz)$
\[
\nabla f(x,y,z) = \answer{(yz\cos(xyz), xz\cos(xyz),xy\cos(xyz))}
\]
$g(x,y) = x^2e^y + y$
\[
\nabla g(x,y) = \answer{(2xe^y, x^2e^y+1)}
\]
$h(x_1,x_2,x_3,x_4) = x_1^2x_2+x_1x_3+x_2x_4$
\[
\nabla h(x_1,x_2,x_3,x_4) = \answer{(2x_1x_2+x_3,x_1^2+x_4,x_1,x_2)}
\]
\end{problem}

\section*{Differentiability}

Now that we've defined the gradient, let's revisit our definition of differentiability for a function from $\mathbb{R}^2$ to $\mathbb{R}$. We used the function
\[
h(x,y) = f(a,b) + f_x(a,b)(x-a)+f_y(a,b)(y-b).
\]
Looking at the terms $f_x(a,b)(x-a)+f_y(a,b)(y-b)$, we can rewrite this as a dot product of two vectors:
\[
f_x(a,b)(x-a)+f_y(a,b)(y-b) = (f_x(a,b), f_y(a,b))\cdot (x-a,y-b).
\]
The first vector is the gradient of $f$ evaluated at $(a,b)$, so we can rewrite this as
\[
(f_x(a,b), f_y(a,b))\cdot (x-a,y-b) = \nabla f(a,b)\cdot (x-a,y-b).
\]
If we take $\vec{x}=(x,y)$ and $\vec{a} = (a,b)$, we can write this as
\[
\nabla f(\vec{a})\cdot (\vec{x}-\vec{a}).
\]
With these notational changes in mind, we now define differentiability for a function $f:\mathbb{R}^n\rightarrow\mathbb{R}$.

\begin{definition}
Consider a function $f:\mathbb{R}^n\rightarrow\mathbb{R}$. For a point $\vec{a}\in\mathbb{R}^n$, define
\[
h(\vec{x}) = f(\vec{a})+\nabla f(\vec{a})\cdot (\vec{x}-\vec{a}).
\]
We say that $f$ is \emph{differentiable} at $\vec{a}$ if
\[
\lim_{\vec{x}\rightarrow \vec{a}}\frac{f(\vec{x}) - h(\vec{x})}{\|\vec{x}-\vec{a}\|} = 0.
\]
If $f$ is differentiable, we say that $h(\vec{x})$ is the \emph{tangent hyperplane} to $f$ at $\vec{a}$.

If any of the partial derivatives of $f$ do not exist, or the above limit does not exist or is not $0$, then $f$ is not differentiable at $\vec{a}$.
\end{definition}

\begin{example}
We'll use this definition of differentiability to prove that the function $f(x,y,z) = xy+z$ is differentiable at $(1,1,1)$.

First, we find the gradient of $f$.
\[
\nabla f(x,y,z) = \answer{(y,x,1)}
\]
At the point $(1,1,1)$, we have
\[
\nabla f(1,1,1) = \answer{(1,1,1)}.
\]
From this, we find the formula for $h(x,y,z)$.
\begin{align*}
h(x,y,z) &= f(1,1,1)+\nabla f(1,1,1)\cdot ((x,y,z)-(1,1,1))\\
&= 2+(1,1,1)\cdot (x-1,y-1,z-1)\\
&= \answer{x+y+z-1}
\end{align*}

Next, we evaluate the limit
\begin{align*}
\lim_{(x,y,z)\rightarrow (1,1,1)}\frac{f(x,y,z) - h(x,y,z)}{\|(x,y,z)-(1,1,1)\|} &= \lim_{(x,y,z)\rightarrow (1,1,1)}\frac{(xy+z) - (x+y+z-1)}{\sqrt{(x-1)^2+(y-1)^2+(z-1)^2}}\\
&= \lim_{(x,y,z)\rightarrow (1,1,1)}\frac{xy-x-y+1}{\sqrt{(x-1)^2+(y-1)^2+(z-1)^2}}.\\
\end{align*}
To evaluate this limit, we switch to translated spherical coordinates
\begin{align*}
x &= 1+\rho\cos\theta\sin\phi,\\
y &= 1+\rho\sin\theta\sin\phi,\\
z &= 1+\rho\cos\phi.
\end{align*}
Making this change, we obtain
\begin{align*}
\lim_{(x,y,z)\rightarrow (1,1,1)}\frac{xy-x-y+1}{\sqrt{(x-1)^2+(y-1)^2+(z-1)^2}} &= \lim_{\rho\rightarrow 0}\frac{(1+\rho\cos\theta\sin\phi)(1+\rho\sin\theta\sin\phi) - (1+\rho\cos\theta\sin\phi) - (1+\rho\sin\theta\sin\phi) + 1}{|\rho|}\\
&= \lim_{\rho\rightarrow 0}\frac{\rho^2\cos\theta\sin\theta\sin^2\phi}{|\rho|}.
\end{align*}
Since $-|\rho|\leq \frac{\rho^2\cos\theta\sin\theta\sin^2\phi}{|\rho|} \leq |\rho|$, we use the squeeze theorem to obtain
\[
\lim_{\rho\rightarrow 0}\frac{\rho^2\cos\theta\sin\theta\sin^2\phi}{|\rho|} =0.
\]
Thus, we have shown that $f(x,y,z) = xy+z$ is differentiable at $(1,1,1)$.
\end{example}

\end{document}