\documentclass{ximera}

\graphicspath{{./graphics/}}

\title{Directional Derivatives}
\author{Melissa Lynn}
\outcome{Understand the idea behind directional derivatives, and use the limit definition to compute them.}

\begin{document}
\begin{abstract}
\end{abstract}
\maketitle

In order to find how a function $f:\mathbb{R}^n\rightarrow\mathbb{R}$ changes with each of the input variables, we defined the partial derivatives of $f$. For example, when $n=2$, we defined the partial derivative of $f$ with respect to $x$ to be
\[
f_x(x,y)=\lim_{h\rightarrow 0}\frac{f(x+h,y)-f(x,y)}{h}.
\]
Here, we thought of $y$ as a constant, which made $f$ only a function of $x$, and reduced us to a single variable derivative. This told us how a small change in $x$ would after the value of $f$, if we kept $y$ constant. In other words, the partial derivatives described how the function $f$ was changing in the positive $x$-direction and in the positive $y$-direction.

\begin{image}
\begin{tikzpicture}
\draw[color = white] (-8,0) -- (8,4);
\fill (0,0) circle(2pt);
\draw[->, dashed] (0,0) -- (3,0);
\draw[->, dashed] (0,0) -- (0,3);
\node[anchor = west] at (3.2,0) {positive $x$-direction};
\node[anchor = south] at (0,3.2) {positive $y$-direction};
\end{tikzpicture}
\end{image}

But what if we want to find how $f$ changes if we change both $x$ and $y$? One possible way to do this would be to increase $x$ and $y$ by the same amount, which would be equivalent to finding how $f$ changes as we increase $x$ and $y$ along the line $y=x$.

\begin{image}
\begin{tikzpicture}
\draw[color = white] (-8,0) -- (8,4);
\fill (0,0) circle(2pt);
\draw[->, dashed] (0,0) -- (3,3);
\node[anchor = west] at (3.2,3) {increasing $x$ and $y$ along $x=y$};
\end{tikzpicture}
\end{image}

Alternatively, we could decrease $y$ by twice as much as $x$. This would be equivalent to finding how $f$ changes as decreasing $x$ and $y$ along the line $y=2x$.

\begin{image}
\begin{tikzpicture}
\draw[color = white] (-8,-4) -- (8,0);
\fill (0,0) circle(2pt);
\draw[->, dashed] (0,0) -- (-1.5,-3);
\node[anchor = east] at (-1.7,-3) {decreasing $x$ and $y$ along $y=2x$};
\end{tikzpicture}
\end{image}

As you can see, there are many different ways that we can change $x$ and $y$, corresponding to different directions in the $xy$-plane. In order to determine how $f$ changes as we move in all of these different directions, we will now define directional derivatives.

\section*{Directional derivatives}

We would like to compute the instantaneous rate of change of a function $f:\mathbb{R}^n\rightarrow\mathbb{R}$ at a point as we move in some given direction in $\mathbb{R}^n$. We will model our definition after partial derivatives and single variable derivatives, and use a unit vector $\vec{v}$ to describe the direction.

\begin{definition}
Consider a function $f:\mathbb{R}^n\rightarrow\mathbb{R}$, a point $\vec{a}\in\mathbb{R}^n$, and a direction given by a unit vector $\vec{v}\in\mathbb{R}^n$. Then we define the \emph{directional derivative} of $f$ at $\vec{a}$ in the direction of $\vec{v}$ to be
\[
D_{\vec{v}}f(\vec{a}) = \lim_{h\rightarrow 0}\frac{f(\vec{a}+h\vec{v})-f(\vec{a})}{h},
\]
provided this limit exists.
\end{definition}

Noticing that by looking at $f(\vec{a}+h\vec{v})$, we are finding the value of $f$ when we move a small distance, $h$, in the direction of $\vec{v}$ from the point $\vec{a}$.

\begin{image}
\begin{tikzpicture}
\fill (0,0) circle(2pt);
\node[anchor = north] at (0,0) {$\vec{a}$};
\draw[->, dashed] (0,0) -- (3,3);
\node[anchor = west] at (3.1,3) {$\vec{v}$};
\draw[->, thick] (0,0) -- (1,1);
\node[anchor = west] at (1.1,1) {$\vec{a} + h\vec{v}$};
\end{tikzpicture}
\end{image}

When computing directional derivatives, it's important to remember that the direction must be given by a \emph{unit} vector. Otherwise, the length of the vector will change the value of the limit above. If you'd like to find a directional derivative in a direction given by a non-unit vector $\vec{w}$, you should normalize $\vec{w}$ to unit length.

\begin{example}
We'll compute the directional derivative of $f(x,y) = x^2y+y^2$ at $\vec{a}=(2,0)$, in the direction of $(3,4)$.

Since $(3,4)$ isn't a unit vector, we need to normalize it. Since $\|(3,4)\| = \sqrt{3^2+4^2}=5$, we'll use the vector $\vec{v}=\left(\frac{3}{5},\frac{4}{5}\right)$ to compute our desired directional derivative.
\begin{align*}
D_{\vec{v}}f(\vec{a}) &= \lim_{h\rightarrow 0}\frac{f(\vec{a}+h\vec{v})-f(\vec{a})}{h}\\
&= \lim_{h\rightarrow 0}\frac{f\left((2,0) + h\left(\frac{3}{5},\frac{4}{5}\right)\right)-f(2,3)}{h}\\
&= \lim_{h\rightarrow 0}\frac{f\left(2+\frac{3}{5}h,\frac{4}{5}h \right)-f(2,0)}{h}\\
&= \lim_{h\rightarrow 0}\frac{\left(2+\frac{3}{5}h\right)^2\cdot \frac{4}{5}h + \left(\frac{4}{5}h\right)^2-0}{h}\\
&= \lim_{h\rightarrow 0}\left(2+\frac{3}{5}h\right)^2\cdot \frac{4}{5} + \left(\frac{4}{5}\right)^2h\\
&= 4\cdot \frac{4}{5}\\
&= \frac{16}{5}.
\end{align*}
\end{example}

Fortunately, we won't always need to resort to evaluating directional derivatives using the limit definition. We'll soon see how we can use the gradient to compute directional derivatives.

\end{document}