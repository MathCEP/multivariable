\documentclass{ximera}

\graphicspath{{./graphics/}}

\title{Introduction to Limits}
\begin{document}
\begin{abstract}
\end{abstract}
\maketitle

We would like to eventually define derivatives and integrals for functions on $\mathbb{R}^n$, but before we do this, we'll need to study limits.

In single variable calculus, we gave a formal, epsilon-delta definition of limits.

\begin{definition}
Let $f:\mathbb{R}\rightarrow\mathbb{R}$ be a function. We write
\[
\lim_{x\rightarrow a} f(x) = L
\]
if for all $\epsilon >0$, there exists some $\delta >0$ such that for all $x$ with $0 < |x-a| < \delta$, we have $|f(x)-L| < \epsilon$.
\end{definition}

The idea here is that if $x$ gets close enough to $a$, then $f(x)$ is guaranteed to get close to $L$. This leads us to our second, informal definition of a limit.

\begin{definition}
(Informal definition) We say that
\[
\lim_{x\rightarrow a} f(x) = L
\]
if $f(x)$ approaches $L$ as $x$ approaches $a$.
\end{definition}

When we think of $x$ approaching $a$ along the number line, $\mathbb{R}$, we can approach $a$ from two directions: left and right.

VISUAL

This lead us to the idea of left and right limits.

When we have a function whose domain is a subset of $\mathbb{R}^n$, there are infinitely many possible ways to approach a point. We can approach a point along infinitely many different lines, and we can also ``zig-zag'' or ``spiral'' into a point.

VISUAL

This makes limits in $\mathbb{R}^n$ particularly challenging!

\section*{Showing that limits do not exist}

For now, we'll work with an informal, intuitive definition of limits of functions $f:\mathbb{R}^n\rightarrow\mathbb{R}$. We'll revisit this definition, and provide a formal, epsilon-delta definition, in a later section. We model our informal definition after our definition from single variable calculus.

\begin{definition}
Let $f:\mathbb{R}^n\rightarrow \mathbb{R}$. We say that
\[
\lim_{\vec{x}\rightarrow\vec{a}}f(\vec{x})=L
\]
if $f(\vec{x})$ gets close to $L$ as $\vec{x}$ gets close to $\vec{a}$.
\end{definition}

Note that since $f$ is a function from $\mathbb{R}^n$ to $\mathbb{R}$, the inputs of $f$ are points/vectors in $\mathbb{R}^n$, and the outputs of $f$ are numbers in $\mathbb{R}$.

An important consequence of this definition is that if we approach the point $\vec{a}$ along any path, the value of the function $f$ should always approach the limit $L$ (if the limit exists). This provides us with an important tool for showing that some limits do not exist.

\begin{proposition}
Consider a function $f:\mathbb{R}^n\rightarrow \mathbb{R}$, and a point $\vec{a}\in\mathbb{R}^n$. Suppose there are continuous paths $vec{x}(t)$ and $\vec{y}(t)$ such that $\vec{x}(t_1) = \vec{y}(t_2) = \vec{a}$, and suppose that
\[
\lim_{t\rightarrow t_1}f(\vec{x}(t))\neq \lim_{t\rightarrow t_2}f(\vec{y}(t))
\]
(or one of these limits does not exist). Then $\lim_{\vec{x}\rightarrow\vec{a}}f(x)$ does not exist.
\end{proposition}

Note that the contrapositive of this statement is false: if we find two paths along which a function has the same limit, this does not guarantee that the overall limit exists.

We will prove this proposition once we give the epsilon-delta definition of a limit, but for now, we'll use it to show that some limits do not exist.

\begin{example}
Consider the function $f(x,y) = \frac{x^2-y^2}{x^2+y^2}$. We will show that $\lim_{(x,y)\rightarrow(0,0)} f(x,y)$ does not exist.

First, let's see what happens when we approach the origin along the $x$-axis. In this case, the $y$-coordinate will always be $0$. More specifically, we approach along the path $\vec{x}(t) = (t,0)$, and let $t\rightarrow 0$. We find the limit of $f$ along this path:
\begin{align*}
\lim_{t\rightarrow 0}f(\vec{x}(t)) &= \lim_{t\rightarrow 0}f(t,0)\\
&= \lim_{t\rightarrow 0}\frac{t^2-0^2}{t^2+0^2}\\
&= \lim_{t\rightarrow 0}1\\
&= 1.
\end{align*}

Next, let's see what happens when we approach the origin along the $y$-axis. That is, we'll consider the path $\vec{y}(t) = (0,t)$, and take $t\rightarrow 0$. We find the limit of $f$ along this path:
\begin{align*}
\lim_{t\rightarrow 0}f(\vec{y}(t)) &= \lim_{t\rightarrow 0}f(0,t)\\
&= \lim_{t\rightarrow 0}\frac{0^2-t^2}{0^2+t^2}\\
&= \answer{-1}.
\end{align*}

Thus, we have found two paths along which $f$ approaches different values. This means that $\lim_{\vec{x}\rightarrow\vec{a}}f(\vec{x})$ does not exist.

We can see this behavior reflected in the graph of $f$.

GRAPH
\end{example}

\begin{example}
Consider the function $f(x,y) = \frac{x^2y}{x^4+y^2}$. We'll investigate whether $\lim_{(x,y)\rightarrow (0,0)}f(x,y)$ exists.

First, let's see what happens when we approach the origin along the $x$- and $y$-axes.

Along the $x$-axis, we use the path $\vec{x}(t) = (t,0)$, and we have
\begin{align*}
\lim_{t\rightarrow 0}f(\vec{x}(t)) &= \lim_{t\rightarrow 0}f(t,0)\\
&= \lim_{t\rightarrow 0}\frac{t^2 0}{t^4+0^2}\\
&= \answer{0}.\\
\end{align*}

Along the $y$-axis, we use the path $\vec{y}(t) = (0,t)$, and we have
\begin{align*}
\lim_{t\rightarrow 0}f(\vec{y}(t)) &= \lim_{t\rightarrow 0}f(0,t)\\
&= \lim_{t\rightarrow 0}\frac{0^2 t}{0^4+t^2}\\
&= \answer{0}.
\end{align*}

Based on these limits, what can we conclude about $\lim_{(x,y)\rightarrow (0,0)}f(x,y)$?
\begin{multipleChoice}
\choice{It exists, and equals $0$.}
\choice{It doesn't exist.}
\choice[correct]{We still don't know if it exists or not.}
\end{multipleChoice}

Next, let's see what happens when we approach the origin along any line $y=mx$. We can parametrize this line as $\vec{z}(t) = (t, mt)$. Along this line, we have
\begin{align*}
\lim_{t\rightarrow 0}f(\vec{z}(t)) &= \lim_{t\rightarrow 0}f(t,mt)\\
&= \lim_{t\rightarrow 0}\frac{t^2 mt}{t^4+(mt)^2}\\
&= \lim_{t\rightarrow 0}\frac{mt^3}{t^4+(mt)^2}\\
&= \lim_{t\rightarrow 0}\frac{mt}{t^2+m^2}\\
&= \answer{0}.
\end{align*}

Based on the above limits, what can we conclude about $\lim_{(x,y)\rightarrow (0,0)}f(x,y)$?
\begin{multipleChoice}
\choice{It exists, and equals $0$.}
\choice{It doesn't exist.}
\choice[correct]{We still don't know if it exists or not.}
\end{multipleChoice}

Finally, let's see what happens when we approach the origin along the parabola $y=x^2$, which can be parametrized as $\vec{w}(t) = (t, t^2)$. Along this path, we have
\begin{align*}
\lim_{t\rightarrow 0}f(\vec{q}(t)) &= \lim_{t\rightarrow 0}f(t,t^2)\\
&= \lim_{t\rightarrow 0}\frac{t^2 t^2}{t^4+(t^2)^2}\\
&= \lim_{t\rightarrow 0}\frac{t^4}{t^4+mt^4}\\
&= \answer{1/2}.
\end{align*}

Based on the above limits, what can we conclude about $\lim_{(x,y)\rightarrow (0,0)}f(x,y)$?
\begin{multipleChoice}
\choice{It exists, and equals $0$.}
\choice[correct]{It doesn't exist.}
\choice{We still don't know if it exists or not.}
\end{multipleChoice}

\end{example}

In the previous example, we saw that we might get the same limit approaching along any line through the origin, but it's still possible that the overall limit might not exist. Thus, we won't be able to show that limits exist by examining specific paths, and we'll need to find other methods to evaluate limits.


\end{document}