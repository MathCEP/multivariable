\documentclass{ximera}

\graphicspath{{./graphics/}}

\title{The Hessian Matrix}
\author{Melissa Lynn}
\outcome{Compute Hessian matrices, and use them to find Taylor polynomials.}

\begin{document}
\begin{abstract}
\end{abstract}
\maketitle

We've been working towards defining some sort of ``second derivative'' for multivariable functions, which can tell us about the second-order behavior of functions. It will also enable us to define degree two Taylor polynomials, and we'll later see how it can be used to classify critical points in optimization. 

\section*{The Hessian Matrix}

Suppose we have a differentiable function $f:\mathbb{R}^n\rightarrow\mathbb{R}$. We can take the gradient of this function.
\[
\nabla f = \left(\frac{\partial f}{\partial x_1}, \frac{\partial f}{\partial x_2}, \frac{\partial f}{\partial x_3},..., \frac{\partial f}{\partial x_n} \right)
\]
We can think of the gradient as a function $\nabla f: \mathbb{R}^n\rightarrow\mathbb{R}^n$. Assuming all of the partial derivatives exist, we can then take the derivative matrix of $\nabla f$. This gives us a square $n\times n$ matrix, which we call the Hessian of $f$.

\begin{definition}
The \emph{Hessian Matrix} of $f:\mathbb{R}^n\rightarrow \mathbb{R}$ is the $n\times n$ matrix
\[
Hf = \begin{pmatrix}
\frac{\partial ^2 f}{\partial x_1^2} & \frac{\partial ^2 f}{\partial x_2\partial x_1} & \cdots & \frac{\partial ^2 f}{\partial x_n\partial x_1}\\
\frac{\partial ^2 f}{\partial x_1x_2} & \frac{\partial ^2 f}{\partial x_2^2} & \cdots & \frac{\partial ^2 f}{\partial x_n\partial x_2}\\
\vdots & \vdots &\ddots & \vdots\\
\frac{\partial ^2 f}{\partial x_1x_n} & \frac{\partial ^2 f}{\partial x_2x_n} & \cdots & \frac{\partial ^2 f}{\partial x_n^2}\\
\end{pmatrix}.
\]
\end{definition}

Notice that if $f$ has continuous first and second order partial derivatives, then the Hessian matrix will be symmetric by Clairaut's Theorem.

\begin{example}
Consider the function $f(x,y) = x + 2xy + 3y^3$. We'll compute the Hessian of $f$. First, we find the gradient of $f$.
\[
\nabla f = \answer{(1+2y, 2x+9y^2)}
\]
Taking the derivative matrix of the gradient, we obtain the Hessian of $f$.
\[
Hf = \begin{pmatrix}
\answer{0} & \answer{2}\\
\answer{2} & \answer{18y}
\end{pmatrix}
\]

Consider the function $g(x,y,z) = e^{xyz}$. We'll compute the Hessian of $g$. First, we find the gradient of $g$.
\[
\nabla g = \answer{(yze^{xyz}, xze^{xyz}, xye^{xyz})}
\]
Taking the derivative matrix of the gradient we obtain the Hessian of $f$.
\[
Hf = \begin{pmatrix}
\answer{y^2z^2e^{xyz}} & \answer{z(xyz+1)e^{xyz}} & \answer{y(xyz+1)e^{xyz}}\\
\answer{z(xyz+1)e^{xyz}} & \answer{x^2z^2e^{xyz}} & \answer{x(xyz+1)e^{xyz}}\\
\answer{y(xyz+1)e^{xyz}} & \answer{x(xyz+1)e^{xyz}} & \answer{x^2y^2e^{xyz}}
\end{pmatrix}
\]
\end{example}

\section*{Taylor Polynomials}

We're now in position to define the second-order Taylor polynomial of a function $f:\mathbb{R}^n\rightarrow \mathbb{R}$, using the Hessian matrix to find the degree two terms.

\begin{definition}
Consider a function $f:\mathbb{R}^n\rightarrow\mathbb{R}$. The degree two Taylor polynomial of $f$ centered at $\vec{a}$ is
\[
p(\vec{x}) = f(\vec{a}) + \nabla f(\vec{a})\cdot (\vec{x}-\vec{a}) + \frac{1}{2} (\vec{x}-\vec{a})^T Hf(\vec{a})(\vec{x}-\vec{a}).
\]
\end{definition}

\begin{example}
We'll compute the degree two Taylor polynomial of the function $f(x,y) = x^2+2xy+3y^3$ centered at $(2,1)$. We previously found that the gradient and Hessian of this function were
\begin{align*}
\nabla f(x,y) &= \answer{(1+2y, 2x+9y^2)},\\
Hf(x,y) &= \begin{pmatrix}
\answer{0} & \answer{2}\\
\answer{2} & \answer{18y}
\end{pmatrix}.
\end{align*}
Plugging in the point $(2,1)$, we have
\begin{align*}
\nabla f(2,1) &= \answer{(3, 13)},\\
Hf(x,y) &= \begin{pmatrix}
\answer{0} & \answer{2}\\
\answer{2} & \answer{18}
\end{pmatrix}.
\end{align*}
Then, the degree 2 Taylor polynomial of $f$ centered at $(2,1)$ is
\begin{align*}
p_2(x,y) &= f(2,1) + \nabla f(2,1)\cdot (x-2,y-1) + \frac{1}{2} (x-2,y-1)^T Hf(2,1)(x-2,y-1)\\
&= 11+ (3,13)\cdot (x-2,y-1) + \frac{1}{2} \begin{pmatrix}x-2 & y-1\end{pmatrix}\begin{pmatrix}
0 & 2\\
2 & 18
\end{pmatrix}\begin{pmatrix}x-2 \\ y-1\end{pmatrix}\\
&= 11 + 3(x-2)+ 13(y-1) + \frac{1}{2}(4(x-2)(y-1)+18(y-1)^2)\\
&= 5 + x -9y + 2xy + 9y^2.
\end{align*}
\end{example}

Notice that the order two part of the Taylor polynomial,
\[
\frac{1}{2} (\vec{x}-\vec{a})^T Hf(\vec{a})(\vec{x}-\vec{a}),
\]
will be a quadratic form when $\vec{a} = \vec{0}$. We will soon make use of our classification of quadratic forms in order to use the Hessian matrix to determine the order two behavior of a function, which will be useful for optimizing multivariable functions.

\end{document}