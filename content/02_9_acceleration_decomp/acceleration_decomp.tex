\documentclass{ximera}

\graphicspath{{./graphics/}}

\title{Decomposition of Acceleration}
\begin{document}
\begin{abstract}
\end{abstract}
\maketitle

Recall our definition of the moving frame:

\begin{definition}
Given a path $\vec{x}(t)$, we define the \emph{moving frame} of the path to be the triple $(\vec{T},\vec{N},\vec{B})$.

$\vec{T}$ is the \emph{unit tangent vector},
\[
\vec{T}(t) = \dfrac{\vec{x}'(t)}{\|\vec{x}'(t)\|}.
\]

$\vec{N}$ is the \emph{unit normal vector},
\[
\vec{N}(t) = \dfrac{\vec{T}'(t)}{\|\vec{T}'(t)\|}.
\]

$\vec{B}$ is the \emph{unit binormal vector},
\[
\vec{B}(t) = \vec{T}(t)\times\vec{N}(t).
\] 

The moving frame is also called the \emph{TNB frame}.
\end{definition}

Throughout our study of paths, we've found a lot of ways that we can describe the behavior of the path, in addition to the moving frame:
\begin{itemize}
\item The velocity vector, $\vec{v}(t)= \vec{x}'(t)$.
\item The speed, $s'(t) = \|\vec{x}'(t)\|$.
\item The acceleration, $\vec{a}(t) = \vec{x}''(t)$.
\item The arclength function, $s(t) = \int_a^t \|\vec{x}'(\tau)\|d\tau$.
\item The parametrization with respect to arclength, $\vec{x}(s)$.
\item The curvature, $\kappa(t) = \|\vec{T}'(s)\| = \frac{\|\vec{T}'(t)\|}{\|\vec{x}'(t)\|}$.
\item The osculating circle and osculating plane (DID I EVER DEFINE THIS?).
\end{itemize}

We'll now explore the connections between these concepts. In particular, we'll derive a useful decomposition of the acceleration vector as a linear combination of the unit tangent and unit normal vectors.

\section*{Curvature and torsion}

We begin with the following result, which connects the curvature and unit normal vector with the derivative of the unit tangent vector with respect to arclength.

\begin{proposition}
\[
\dfrac{d\vec{T}}{ds} = \kappa\vec{N}
\]
\end{proposition}

\begin{proof}
First, thinking of arclength $s$ as a function of $t$ and using the chain rule, we have
\[
\dfrac{d}{dt}\vec{T}(s(t)) = \vec{T}'(s(t))s'(t).
\]
Recognizing $s'(t)$ as the speed, we can rewrite this as 
\[
\dfrac{d}{dt}\vec{T}(t) = \vec{T}'(s)\|\vec{x}'(t)\|.
\]
Solving for $\vec{T}'(s)$, we have 
\[
\dfrac{d\vec{T}}{ds} = \dfrac{\vec{T}'(t)}{\|\vec{x}'(t)\|}.
\]

Turning to the other side of the equality, recall that we defined the curvature to be $\kappa(t) = \|\vec{T}'(s)\| $, and we found that we could also compute this as $\frac{\|\vec{T}'(t)\|}{\|\vec{x}'(t)\|}$. We defined the unit normal vector to be $\dfrac{\vec{T}'(t)}{\|\vec{T}'(t)\|}$.

Putting all of this together, we have
\begin{align*}
\kappa(t)\vec{N}(t) &= \frac{\|\vec{T}'(t)\|}{\|\vec{x}'(t)\|}\dfrac{\vec{T}'(t)}{\|\vec{T}'(t)\|}\\
&= \frac{\vec{T}'(t)}{\|\vec{x}'(t)\|}\\
&= \dfrac{d\vec{T}}{ds},
\end{align*}
proving our result.
\end{proof}

It turns out that there is a similar result relating the normal vector with the derivative of the unit binormal vector with respect to arclength:
\[
\dfrac{d\vec{B}}{ds} = \tau\vec{N}.
\]
We haven't talked about the coefficient $\tau$ yet, but this is another important property of curves, called \emph{torsion}.

Together, the curvature and torsion carry a lot of important information about the curve. In fact, the curvature and torsion completely determine the curve!

\section*{Decomposition of acceleration}

Recall that the unit tangent vector $\vec{T}$ points in the direction of instantaneous motion, and the unit normal vector $\vec{N}$ points in the direction that a path is turning. So, it shouldn't be too surprising that the acceleration vector is always a linear combination of $\vec{T}$ and $\vec{N}$. However, it's very surprising that we can recognize the coefficients in terms of things we've seen before!

\begin{proposition}
Let $\vec{x}(t)$ be a $C^2$ path in $\mathbb{R}^3$. Then the acceleration vector can be written as
\[
\vec{a} = s''\vec{T} + (s')^2\kappa\vec{N},
\]
where $s$ is the arclength function (so $s'$ is the speed), $\vec{T}$ is the unit tangent vector, $\kappa$ is the curvature, and $\vec{N}$ is the unit normal vector.
\end{proposition} 

\begin{proof}
We begin with some observations.

Recall that we defined the unit tangent vector as $\vec{T}$ as $\vec{T}=\frac{\vec{v}}{\|\vec{v}\|}$, where $\vec{v}(t) = \vec{x}'(t)$ is the velocity vector. Replacing the speed $\|\vec{v}\|$ with $s'$, this gives us
\[
\vec{T} = s'\vec{v}.
\]

Similarly, we defined the unit normal vector as $\vec{N}=\frac{\vec{T}'}{\|\vec{T}'\|}$, so we can write
\[
\vec{T}' = \|\vec{T}'\|\vec{N}.
\]
We have that the curvature is $\kappa = \frac{\|\vec{T}'\|}{s'}$, so we can rewrite this as
\begin{align*}
\vec{T}' &= \|\vec{T}'\|\vec{N}\\
&= (s'\kappa)\vec{N}.
\end{align*}

Finally, we turn to our acceleration vector. We recall that acceleration is the derivative of velocity, and use the product rule with the above observations to obtain
\begin{align*}
\vec{a} &= \vec{v'}\\
&= \frac{d}{dt}(s'\vec{T})\\
&= s''\vec{T} + s'\vec{T}'\\
&= s''\vec{T} + s'(s'\kappa)\vec{N}\\
&= s''\vec{T} + (s')^2\kappa\vec{N}.
\end{align*}
\end{proof}

Immediately from this result, we can make a lot of important observations.

\begin{proposition}
Let $\vec{x}(t)$ be a $C^2$ path in $\mathbb{R}^3$. Then:
\begin{itemize}
\item $\vec{a}$ is a linear combination of $\vec{T}$ and $\vec{N}$.
\item $\vec{a}$ is always in the osculating plane.
\item Since $(s')^2\geq 0$ and $\kappa\geq 0$, the acceleration $\vec{a}$ points in the direction that we're turning.
\item If $\kappa = 0$, then $\vec{a}$ is parallel to $\vec{T}$.
\item If speed is constant, then $s''=0$, so $\vec{a}$ is parallel to $\vec{N}$.
\end{itemize}
\end{proposition}

We leave the proofs of these facts as an exercise.

\section*{Summary of notation for parametric curves}

There are a lot of symbols to keep track of when studying the geometry of parametric curves.  To make matters worse, most of them have multiple names.  For example, the derivative of $\vec{x}(t)$ can be denoted by either $\vec{x}'(t)$ or $\dot{\mathbf{x}}(x)$, but we often call it $\vec{v}(t)$ because it represents velocity.  Given a parametrization $\vec{x}(t)$, $t \in [a,b]$, which represents motion of a particle along a curve $C$, we list most of the related functions and their interpretations.\\

\textbf{Position vector}
\begin{itemize}
\item Notation: $ \vec{x}(t)$, $\mathbf{x}(t)$ 
\item Represents the position of a particle at time $t$
\end{itemize}

\textbf{Tangent vector}
\begin{itemize}
\item Notation: $\vec{x}'(t)$, $\dot{\mathbf{x}}(t)$, $\vec{v}(t)$, $\mathbf{x}'(t)$, $\mathbf{v}(t)$ 
\item Derivative of $\vec{x}(t)$; also called the velocity vector.  
\item Its direction shows the direction of instantaneous motion, and its length ($||\vec{v}(t)||=||\vec{x}'(t)||$ etc.) is the instantaneous speed. 
\end{itemize}

\textbf{Unit tangent vector}
\begin{itemize}
\item Notation: $\vec{T}(t)$%, $T(t)$ 
\item Computed as $\vec{x}'(t)/||\vec{x}'(t)||$, $\vec{v}(t)/||\vec{v}(t)||$, etc.
\end{itemize}

\textbf{Derivative of the unit tangent vector}
\begin{itemize}
\item Notation: $\vec{T}'(t)$, $d\vec{T}/dt$%, $T'(t)$, $dT/dt$ 
\item Derivative of unit tangent vector with respect to time.  
\item Not necessarily a unit vector.  
\item Must be perpendicular to $\vec{T}(t)$:
because $\vec{T}$ is a unit vector, $\vec{T}\cdot \vec{T}=1$; differentiating both sides with respect to $t$ gives $2\vec{T}\cdot\vec{T}' = 0$.
\end{itemize}

\textbf{Unit normal vector}
\begin{itemize}
\item Notation: $\vec{N}(t)$%, $N(t)$ 
\item Computed as $\vec{T}'(t) / ||\vec{T}'(t)||$.  
\item Perpendicular to $\vec{T}$, since it's a scaled version of $\vec{T}'$. 
\end{itemize}

\textbf{Binormal vector}
\begin{itemize}
\item Notation: $\vec{B}(t)$%, $B(t)$ 
\item Computed as $\vec{B} = \vec{T} \times \vec {N}$.  
\item Is a unit vector, since the angle between $\vec{T}$ and $\vec{N}$ is $\theta=\pi/2$ and therefore $ ||\vec{T} \times \vec{N}|| = ||\vec{T}|| \, ||\vec{N}|| \, \sin \theta = 1 $.
\end{itemize}


\textbf{Distance Traveled}
\begin{itemize}
\item Notation: $s$, $s(t)$ 
\item Written as $s$ when it's treated as a variable.  
\item We often view $s$ as a function of time, and compute the arclength function
$ s(t) = \int_a^t ||\vec{x}'(u)|| \, du $.
\end{itemize}


\textbf{Speed}
\begin{itemize}
\item Notation: $ds/dt$, $s'(t)$
\item Computed as $ds/dt = s'(t) = ||\vec{x}'(t)||=||\vec{v}(t)||$ as proven by applying the Fundamental Theorem of Calculus to the definition of $s(t)$.
\end{itemize}

\textbf{Curvature}
\begin{itemize}
\item Notation: $\kappa$
\item Measures how quickly a curve ``turns'' at a given point: $\kappa = \left|| \frac{d\vec{T}}{ds} \right||$.  
\item Using the chain rule we can write $\kappa$ as a function of $t$:

\[ \kappa(t) = \left| \left| \frac{d\vec{T}}{ds} \right| \right| = \left| \left| \frac{d\vec{T}}{dt} \frac{dt}{ds} \right| \right| = 
\left| \left| \frac{d\vec{T}/dt}{ds/dt} \right| \right| = \frac{||\vec{T}'(t)||}{s'(t)} = \frac{||\vec{T}'(t)||}{||\vec{x}'(t)||} \]
\item Many other formulas exist, e.g. for curves which are the graphs of functions $y=f(x)$.
\end{itemize}

\textbf{Acceleration vector}
\begin{itemize}
\item Notation: $\vec{x}''(t)$, $\ddot{\mathbf{x}}(t)$, $\vec{x}''(t)$, $\vec{a}(t)$, $\vec{a}(t)$ 
\item Second derivative of $\vec{x}(t)$ and first derivative of velocity.  
\item If we rewrite $\vec{T}(t) = \vec{v}(t)/||\vec{v}(t)|| = \vec{v}(t)/s'(t)$ as $\vec{v}=s'\vec{T}$, we can apply the product rule to calculate:
\[ \vec{a}(t) = \color{red} s''(t) \color{black}\vec{T}(t) + \color{red} s'(t) \color{black} \vec{T}'(t) 
% \text{ or } \vec{a} 
= \color{red} s'' \color{black} \vec{T} + \color{red} s'||\vec{T}'|| \color{black}\vec{N}
= \color{red} s'' \color{black} \vec{T} + \color{red} (s')^2 \kappa \color{black}\vec{N} \]
The second equation comes from rewriting $\vec{N}(t) = \vec{T}'(t) / ||\vec{T}'(t)||$ as $\vec{T}' = ||\vec{T}'||\vec{N}$ and substituting.  The third equation used the fact that $\kappa = ||\vec{T'}||/s'$, so that $||\vec{T}'||=\kappa s'.$  We've made the scalar functions red and the vectors black to emphasize that acceleration is a linear combination of $\vec{T}$ and $\vec{N}$ (or $\vec{T}'$).
\end{itemize}


I COPIED THIS FROM THE HANDOUT (except for formatting, some notation, and a little rewriting)... HOPEFULLY THAT'S FINE?

SOME SORT OF RANDOMIZED NOTATION QUIZ?

\end{document}