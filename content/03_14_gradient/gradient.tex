\documentclass{ximera}

\graphicspath{{./graphics/}{./content/03_14_gradient/graphics/}}

\title{The Gradient and Level Sets}
\author{Melissa Lynn}
\outcome{Understand the relationship between the gradient and directional derivatives. Use the gradient to compute directional derivatives and identify the direction of greatest increase.}

\begin{document}
\begin{abstract}
\end{abstract}
\maketitle

We've defined the directional derivatives of a function, which allow us to determine how a function is changing in various directions.

\begin{definition}
Consider a function $f:\mathbb{R}^n\rightarrow\mathbb{R}$, a point $\vec{a}\in\mathbb{R}^n$, and a direction given by a unit vector $\vec{v}\in\mathbb{R}^n$. Then we define the \emph{directional derivative} of $f$ at $\vec{a}$ in the direction of $\vec{v}$ to be
\[
D_{\vec{v}}f(\vec{a}) = \lim_{h\rightarrow 0}\frac{f(\vec{a}+h\vec{v})-f(\vec{a})}{h},
\]
provided this limit exists.
\end{definition}

However, we would like an easier way to evaluate directional derivatives, that doesn't require the limit definition.

Such a method will require use of the gradient of the function. Recall our definition of the gradient of a function $f:\mathbb{R}^n\rightarrow\mathbb{R}$.

\begin{definition}
Consider a function $f:\mathbb{R}^n\rightarrow\mathbb{R}$. The \emph{gradient} of $f$ is the function $\nabla f:\mathbb{R}^n\rightarrow\mathbb{R}^n$, defined by
\[
\nabla f = \left(\frac{\partial f}{\partial x_1},\frac{\partial f}{\partial x_2},...,\frac{\partial f}{\partial x_n}\right).
\]
\end{definition}

We'll see that this vector turns out to be closely related to directional derivatives.

\section*{The gradient and directional derivatives}

Let's suppose we have a differentiable function $f:\mathbb{R}^n\rightarrow\mathbb{R}$, and consider our definition of the directional derivative a function $f$ at $\vec{a}$ in the direction of a unit vector $\vec{v}$, which was
\[
D_{\vec{v}}f(\vec{a}) = \lim_{h\rightarrow 0}\frac{f(\vec{a}+h\vec{v})-f(\vec{a})}{h}.
\]
We'll rewrite this definition by considering another function, $F(h) = f(\vec{a}+h\vec{v})$. Notice that $F$ is a single variable function, and when we rewrite the directional derivative, we have
\begin{align*}
D_{\vec{v}}f(\vec{a})& = \lim_{h\rightarrow 0}\frac{f(\vec{a}+h\vec{v})-f(\vec{a})}{h}\\
&= \lim_{h\rightarrow 0 }\frac{F(h)-F(0)}{h-0}\\
&= F'(0).
\end{align*}
So, the directional derivative is just the derivative of this single variable function $F(h)$ evaluated at $0$. Revisiting our definition of $F(h)$, we can use the chain rule to find the derivative of $F$.
\begin{align*}
\frac{d}{dh}F(h) &= \nabla f(\vec{a}+h\vec{v})\cdot\frac{d}{dh}(\vec{a}+h\vec{v})\\
&= \nabla f(\vec{a}+h\vec{v})\cdot\vec{v}\\
\end{align*}
Evaluating at $h=0$, we have
\begin{align*}
D_{\vec{v}}f(\vec{a}) &= F'(0)\\
&= \nabla f(\vec{a})\cdot \vec{v}.
\end{align*}

Thus, we have arrived at the following result.

\begin{theorem}
Suppose $f:\mathbb{R}^n\rightarrow\mathbb{R}$ is differentiable at $\vec{a}\in\mathbb{R}^n$. Then $D_{\vec{v}}f(\vec{a})$ exists for all unit vectors $\vec{v}\in\mathbb{R}^n$, and 
\[
D_{\vec{v}}f(\vec{a}) = \nabla f(\vec{a})\cdot \vec{v}.
\]
\end{theorem}

Let's use this result to compute some directional derivatives.

\begin{example}
We'll compute the directional derivative of $f(x,y) = x^2y+y^2$ at $\vec{a}=(2,0)$, in the direction of $(3,4)$. (We previously computed this directional derivative using the limit definition.)

Since $(3,4)$ isn't a unit vector, we need to normalize it. Since $\|(3,4)\| = \sqrt{3^2+4^2}=5$, we'll use the vector $\vec{v}=\left(\frac{3}{5},\frac{4}{5}\right)$ to compute our desired directional derivative.

Next, we'll need the gradient of $f$.
\[
\nabla f(x,y) = (2xy, x^2+2y)
\]
Since the partial derivatives of $f$ are polynomials, they are continuous, so $f$ is differentiable. Thus, we can use the above theorem to compute the directional derivative.

Then, we can compute the directional derivative as
\begin{align*}
D_{\vec{v}}f(\vec{a}) &= \nabla f(\vec{a})\cdot \vec{v}\\
&= \nabla f(2,0)\cdot \left(\frac{3}{5},\frac{4}{5}\right)\\
&= (0,4)\cdot \left(\frac{3}{5},\frac{4}{5}\right)\\
&= \frac{16}{5}.
\end{align*}

This matches what we had previously computed using the definition of directional derivatives.
\end{example}

\begin{problem}
Compute the directional derivative of $f(x,y,z) = 3xy+xz^2$ at $\vec{a}=(2,0,1)$, in the direction of $(2, 2, 1)$.
\[
D_{\vec{v}}f(\vec{a}) = \answer{6}
\]

Compute the directional derivative of $f(x,y,z) = 3xy+xz^2$ at $\vec{a}=(2,0,1)$, in the direction of $(-2,1,-1)$.
\[
D_{\vec{v}}f(\vec{a}) = \answer{0}
\]
\end{problem}

\section*{The gradient and level sets}

We've shown that for a differentiable function $f:\mathbb{R}^n\rightarrow\mathbb{R}$, we can compute directional derivatives as
\[
D_{\vec{v}}f(\vec{a}) = \nabla f(\vec{a})\cdot \vec{v}.
\]
What does this mean for the possible values for a directional derivative? Recall that the dot product can be computed as
\[
\nabla f(\vec{a})\cdot \vec{v} = \|\nabla f(\vec{a})\|\;\|\vec{v}\|\cos\theta,
\]
where $\theta$ is the angle between the two vectors. Since $\vec{v}$ is a unit vector, we have
\[
\|\nabla f(\vec{a})\|\;\|\vec{v}\|\cos\theta = \|\nabla f(\vec{a})\|\cos\theta.
\]
Since $-1\leq \cos\theta \leq 1$, we have that
\[
-\|\nabla f(\vec{a})\|\leq D_{\vec{v}}f(\vec{a})\leq \|\nabla f(\vec{a})\|.
\]
In particular, the largest that $D_{\vec{v}}f(\vec{a})$ can be is $\|\nabla f(\vec{a})\|$, and this occurs when $\vec{v}$ points in the same direction as $\nabla f(\vec{a})$, so that $\theta = 0$. Thus, the gradient points in the direction of greatest increase.

On the other hand, the minimum value that $D_{\vec{v}}f(\vec{a})$ can have is $-\|\nabla f(\vec{a})\|$, and this occurs when $\vec{v}$ points in the opposite direction from $\nabla f(\vec{a})$, in the direction of $-\nabla f(\vec{a})$. Thus, $-\nabla f(\vec{a})$ points in the direction of greatest decrease.

Additionally, from $D_{\vec{v}}f(\vec{a}) = \nabla f(\vec{a})\cdot \vec{v}$, we can see that  $\vec{v}$ is perpendicular to $\nabla f(\vec{a})$ if and only if $D_{\vec{v}}f(\vec{a})=0$. But what does it mean for $D_{\vec{v}}f(\vec{a})$? This means that there is no instantaneous change in $f$ in the direction of $\vec{v}$, which means that $\vec{v}$ will be a tangent vector to a level curve.

\begin{example}
Consider the graph of the function $f(x,y) = x^2+y^2$, which is a paraboloid.

\begin{image}
\includegraphics[width = \textwidth]{CalcPlot3d-paraboloid}
\end{image}

Let's consider how this function changes near the point $(2,1)$.

The gradient of $f$ at $(2,1)$ is
\[
\nabla f(2,1) = \answer{(4,2)}.
\]
Consulting the graph of $f$ near the point $(2,1)$, we can confirm that it increases most rapidly when we move in the direction of $\nabla f(2,1)$. We can also confirm that it decreases most rapidly when we move in the direction of $-\nabla f(2,1)$.

The point $(2,1)$ lies on the level curve $x^2 + y^2 = 5$. A tangent vector to this level curve at the point $(2,1)$ is given by $(-1, 2)$.

\begin{image}
\begin{tikzpicture}
\draw[<->] (-4,0) -- (4,0);
\node[anchor = west] at (4,0) {$x$};
\draw[<->] (0,-4) -- (0,4);
\node[anchor = south] at (0,4) {$y$};
\draw (0,0) circle (2.236);
\fill (2,1) circle(2pt);
\node[anchor = west] at (2,1) {$(2,1)$};
\draw[->] (2,1) -- (1, 3);
\node[anchor = south] at (1,3) {$(-1,2)$};
\end{tikzpicture}
\end{image}

This vector, $(-1,2)$, is perpendicular to our gradient $\nabla f(2,1)$.
\end{example}

We'll state this observation more formally, and prove that the gradient is perpendicular to the level curves.

\begin{theorem}
Consider a function $f:\mathbb{R}^n\rightarrow\mathbb{R}$, and suppose $f$ is of class $\mathcal{C}^1$. For some constant $c$, consider the level set
\[
S = \{\vec{x}\in\mathbb{R}^n\;:\;f(\vec{x})=c\}.
\]
Then, for any point $\vec{x}_0$ in $S$, the gradient $\nabla f(\vec{x}_0)$ is perpendicular to $S$.
\end{theorem} 

\begin{proof}
We need to show that for any vector $\vec{a}$ which is tangent to $S$ at $\vec{x}_0$, we have that $\vec{a}$ is perpendicular to $\nabla f(\vec{x}_0)$.

If $\vec{a}$ is tangent to $S$, we can find a parametrized curve $\vec{x}(t)$ lying in $S$ such that $\vec{x}_0=\vec{x}(t_0)$ and $\vec{x}'(t_0) = \vec{a}$. We will show that $\nabla f(\vec{x}_0)$ is perpendicular to $\vec{a} = \vec{x}'(t_0)$.

By the definition of $S$, and since $\vec{x}(t)$ lies in $S$, 
\[
f(\vec{x}(t))=c
\]
for all $t$. Differentiating both sides of this identity, and using the chain rule on the left side, we obtain
\[
\nabla f(\vec{x}(t))\cdot \vec{x}'(t) = 0.
\]
Plugging in $t=t_0$, this gives us
\[
\nabla f(\vec{x}(t_0))\cdot \vec{x}'(t_0) = 0,
\]
which we can rewrite as
\[
\nabla f(\vec{x}_0)\cdot \vec{x}'(t_0) = 0.
\]
Thus, we have shown that $\nabla f(\vec{x}_0)$ is perpendicular to the level set $S$.
\end{proof}

\end{document}