\documentclass{ximera}
\graphicspath{{./content/hw_08_differentiability/graphics/}{./graphics/}}
\title{Homework 8: Differentiability}
\begin{document}
\begin{abstract}
\end{abstract}
\maketitle

\section*{Graded Problems}

\begin{problem} Consider the linear transformation $\mathbf{T}:\mathbb{R}^3\rightarrow\mathbb{R}^2$ given by
\[
T(x,y,z) = \left(\begin{array}{ccc}
1&0&2\\
4&1&2
\end{array}\right)\left(\begin{array}{c}
x\\
y\\
z
\end{array}\right).
\]
\begin{enumerate}
\item Compute $D\mathbf{T}(0,0,0)$.
\item Explain why your answer to (a) makes sense in the context of linear approximations.
\end{enumerate}
\end{problem}

\begin{problem} Let $\mathbf{f}:\mathbb{R}^{2} \to \mathbb{R}^{2}$ be the function given by \[ \mathbf{f}(x,y)=(x^{2}+y, 3y).\]  In this problem you will prove according to the limit definition that $\mathbf{f}$ is differentiable at the point $\mathbf{a}=(1,2)$. \emph{In each part, do all calculations by hand and show appropriate amounts of work. Do not use technology.}
\begin{enumerate}
\item Find the derivative matrix $D\mathbf{f}$ of $\mathbf{f}$.

\item To prove $\mathbf{f}$ is differentiable, you must show it has a good linear approximation at the point $\mathbf{a}=(1,2)$.  Using your answer from part (a), determine the formula for this linear approximation.  (No simplifications necessary.)

\item Use Definition~3.8 to prove that $\mathbf{f}$ is differentiable at $\mathbf{a}=(1,2)$.  In this step it would be very useful to simplify the numerator of the ``difference quotient'' as much as possible, and then make some substitutions on the top and bottom to help you evaluate the limit.
\end{enumerate}
\end{problem}

\section*{Professional Problem}

\begin{problem} The first draft of your project is due this week.
\end{problem}

\section*{Completion Packet}

\begin{problem}

Find the gradient of each function at the given point.

\begin{enumerate}

\item $f(x,y) = xe^{xy}+x^2y$, at the point $(1,1)$.

\item $g(x,y) = \sin(x^2+y)+y\cos(x)$, at the point $(0, \pi)$.

\end{enumerate}
\end{problem}

\begin{problem}
Find the matrix of partial derivatives of the following functions.

\begin{enumerate}

\item $f(x,y,z) = \ln(x^2y) + xyz$

\item $\mathbf{f}(x,y,z) = \left(\frac{\sqrt{x^2+y^2}}{z}, z^2y^3\right)$

\item $\mathbf{f}(x) = (\ln(x), xe^x, \sin(x))$

\end{enumerate}
\end{problem}

\begin{problem}
 Consider the function $\mathbf{f}(x,y,z) = \left(\frac{1}{x^2+y^2}, \frac{xy}{z}\right)$. \begin{enumerate}
\item Compute the partial derivatives of $\mathbf{f}$.
\item Show that $\mathbf{f}$ is differentiable on its domain.
\end{enumerate}
\end{problem}

\begin{problem}
Find the point where the plane tangent to $z = x^2+y^2$ at $(1,1,2)$ intersects the $z$-axis.
\end{problem}

\begin{problem}
Because the partial derivatives of a function are necessary to construct the limit in the definition of differentiability, a sort of converse to Theorem~3.10 is: \textit{ If $\mathbf{f}: D \subset \mathbb{R}^{n} \to \mathbb{R}^{m}$ is differentiable at $\mathbf{a} \in \mathbb{R}^{n}$, then the partial derivatives of $\mathbf{f}(\mathbf{x})$ must all be defined at $\mathbf{a}$. }  Use this result to show the following functions are not differentiable at the indicated point.

\begin{enumerate}

\item $f(x,y,z) = \sqrt{x^{2}+y^{2}+z^2}$, at the point $(0,0,0)$.\\

\item $g(x,y) = |x-y|$, at the point $(3,3)$.\\

\end{enumerate}

\end{problem}

\end{document}
