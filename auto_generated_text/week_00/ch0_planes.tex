\documentclass{ximera}  

\title{Representations of Lines and Planes}  

\begin{document}  
\begin{abstract}  
%abstract
\end{abstract}  
\maketitle 

In this section, we review the different ways we can represent lines and planes, including parametric representations.

\section{Representations of Lines}

When you think of describing a line algebraically, you might think of the standard form
\[
y = mx + b,
\]
where $m$ is the slope and $b$ is the $y$-intercept. This is often called \emph{slope-intercept} form.

 In addition to slope-intercept form, there are several other ways to represent lines. For example, you may remember using \emph{point-slope} form in single variable calculus. We can describe a line of slope $m$ going through a point $(x_0, y_0)$ with the equation
 \[
 y-y_0 = m(x-x_0).
 \]
 It's important to note that there are many different possible choices for the point $(x_0,y_0)$. Because of this, unlike slope-intercept form, point-slope form does not give a unique representation of a line.
 
 In linear algebra, we saw that we could parametrize a line using a vector $\vec{v}=(v_1,v_2)$ giving the direction of the line, and a point $(x_0, y_0)$ that the line passes through. We parametrize the line as
 \begin{align*}
 \vec{x}(t) &= (v_1,v_2)t+ (x_0, y_0),\\
 &= (v_1t+x_0, v_2t+y_0).
 \end{align*}
 Note that this representation works a bit differently from the previous two representations. In slope-intercept form and point-slope form, the line was the set of points $(x,y)$ satisfying the given equation. However, in the parametrization, we plug in values for the parameter $t$ in order to get points on the line.
 
 Unlike slope-intercept form and point-slope form, the parametrization of a line can easily be generalized to three or more dimensions. That is, a line in $\mathbb{R}^n$ through the point $\vec{a}$ and in the direction of the vector $\vec{v}$ can be parametrized as
 \[
 \vec{x}(t) = \vec{v}t + \vec{a},
 \]
 for $t\in\mathbb{R}$. 
 
 If we would like to describe a line in higher dimensions using equations (rather than a parametrization), we would need more than one equation. For example, in $\mathbb{R}^3$, we would require two equations to determine a line.
 
 \section{Representations of Planes}
 
 We also have multiple ways to represent planes. Here, we'll focus on planes in $\mathbb{R}^3$.
 
 Recall that a plane can be determined by two vectors (giving the ``direction'' of the plane) and a point that the plane passes through. We can use this to give a parametrization for the plane through the point $\vec{a}$ and parallel to the vectors $\vec{v}$ and $\vec{w}$:
 \[
 \vec{x}(s,t) = \vec{v}s+\vec{w}t + \vec{a},
 \]
 for $s$ and $t$ in $\mathbb{R}$. Note that we require two parameters for the parametrization of the plane.

We can also describe a plane using a single linear equation in $x$, $y$, and $z$. For example,
\[
2x+4y-z=9
\]
defines a plane. A standard way to do this is using a point on the plane and a normal vector to the plane. Recall that a normal vector is perpendicular to every vector in the plane. If $\vec{n}=(n_1,n_2,n_3)$ is a normal vector to a plane passing through the point $\vec{a}=(a_1,a_2,a_3)$, the plane is defined by the equation
\[
\vec{n}\cdot(\vec{x}-\vec{a})=0.
\]
This can be rewritten as
\[
n_1(x-a_1)+n_2(y-a_2)+n_3(z-a_3)=0.
\]

\section{Summary}

We reviewed various representations of lines and planes, including parametrizations.

\end{document}