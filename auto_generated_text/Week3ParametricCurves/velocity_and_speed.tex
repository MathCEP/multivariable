\documentclass{ximera}

\graphicspath{{./auto_generated_text/Week3ParametricCurves/graphics/}{./graphics/}}

\title{Velocity and Speed}
\begin{document}
\begin{abstract}
\end{abstract}
\maketitle

In this activity, we learn how to find the velocity and speed of a parametrized curve in $\mathbb{R}^n$.

\section{Derivatives}

Consider a path $\vec{x}:I\subset\mathbb{R}\rightarrow\mathbb{R}^n$ which parametrizes a curve in $\mathbb{R}^n$. We often think about this as a particle tracing out the curve as time, given by $t$, passes. We would like to be able to understand and describe the motion of the particle on the curve, and find its velocity and speed, in particular. In order to do this, we need to figure out how to differentiate a path.

Before we define the derivative of a path, we quickly review the single variable definition of a derivative, given in Calculus I.

Given a single variable function $f(x)$, we found the instantaneous rate of change at $x$ of this function by taking the derivative of $f$ at $x$. The derivative also told us the slope of the tangent line at $x$. In order to compute this, we imagined finding the slope of secant lines getting closer and closer to the point. Taking a limit, we obtained the slope of the tangent line.

PICTURE

The slope of the secant line through the points $(x,f(x))$ and $(x+h,f(x+h))$ is given by $\dfrac{f(x+h)-f(x)}{h}$, so we defined the derivative of $f$ at $x$ to be
\[
f'(x) = \lim_{h\rightarrow 0}\frac{f(x+h)-f(x)}{h}.
\]

We use the same idea for a path $\vec{x}$ in $\mathbb{R}^n$. We consider secant vectors from $\vec{x}(t)$ to $\vec{x}(t+h)$ as $h\rightarrow 0$.

PICTURE

Scaling these vectors to account for the change in the parameter and taking a limit, we arrive at the definition of the derivative.

\begin{definition}
Let $\vec{x}:I\subset\mathbb{R}\rightarrow\mathbb{R}^n$ be a path in $\mathbb{R}^n$. We define the derivative of $\vec{x}$ at $t$ to be
\[
\vec{x}'(t) = \lim_{h\rightarrow 0} \frac{\vec{x}(t+h) - \vec{x}(t)}{h},
\]
if the limit exists.

We also call $\vec{x}'(t)$ the \emph{velocity vector} of $\vec{x}$, and write it as $\vec{v}(t)$.
\end{definition}

When we first defined derivatives in Calculus I, we spent weeks figuring out how to compute them. We started computing using only the limit definition, then we introduced the power rule, the product rule, the chain rule, and so on. Fortunately, we don't need to repeat this process in Multivariable Calculus: we can take advantage of our previous experience computing derivatives. In order to see why this is the case, let's take another look at our definition for the derivative of a path.

We have $\vec{x}'(t) = \lim_{h\rightarrow 0} \frac{\vec{x}(t+h) - \vec{x}(t)}{h}$ for a path $\vec{x}$. We can write out the path $\vec{x}$ in terms of its components, so
\[
\vec{x}(t) = (x_1(t),...,x_n(t)).
\]
Substituting this into the limit, we have
\begin{align*}
\vec{x}'(t) &= \lim_{h\rightarrow 0} \frac{\vec{x}(t+h) - \vec{x}(t)}{h},\\
&= \lim_{h\rightarrow 0}\frac{(x_1(t+h),...,x_n(t+h)) - (x_1(t),...,x_n(t))}{h},\\
&= \lim_{h\rightarrow 0}\frac{(x_1(t+h) - x_1(t),...,x_n(t+h) - x_1(t))}{h}.
\end{align*}
Dividing through by the scalar $h$ and bringing the limit inside of the vector, we have 
\begin{align*}
\vec{x}'(t) &= \lim_{h\rightarrow 0}\left(\frac{x_1(t+h) - x_1(t)}{h},...,\frac{x_n(t+h) - x_1(t)}{h}\right),\\
&= \left(\lim_{h\rightarrow 0}\frac{x_1(t+h) - x_1(t)}{h},...,\lim_{h\rightarrow 0}\frac{x_n(t+h) - x_1(t)}{h}\right).
\end{align*}
At this point you should be somewhat skeptical. We haven't defined limits of vectors, much less described how to manipulate them. We'll come back to this in a few weeks in much more detail. For now, hopefully it makes sense that looking at what a vector approaches depends on what its components approach, and you'll allow us this sleight of hand.

Looking at the limits inside of the components, they should look familiar. They're derivatives of single variable functions! That is, we now have 
\[
\vec{x}'(t) = (x_1'(t),...,x_n'(t)).
\]
This means that we can differentiate a path by differentiating its components, thus taking advantage of our knowledge of single variable derivatives.

\begin{proposition}
We can differentiate a path by differentiating its components. That is,
\[
\vec{x}'(t) = (x_1'(t),...,x_n'(t)).
\]
\end{proposition}

\begin{example}
Consider the path $\vec{x}(t) = (\cos(t), \sin(t))$ for $0\leq t\leq 2\pi$, which parametrizes the unit circle in $\mathbb{R}^2$. We compute the derivative of this path,
\begin{align*}
\vec{x}'(t) &= \left(\frac{d}{dt}\cos(t), \frac{d}{dt}\sin(t)\right),
&= \left(-\sin(t), \cos(t)\right).
\end{align*}

Consider the path $\vec{y}(t) = (t^2,t^3)$ for $0\leq t\leq 1$.
\[
y'(t) = \answer{(2t, 3t^2)}
\]

Consider the path $\vec{z}(t) = \left(t, e^{t^2}\right)$ for $-\infty < t < \infty$.
\[
z'(t) = \answer{(1, 2te^{t^2})}
\]
\end{example}

\section{Velocity and Speed}

We defined the derivative $\vec{x}'$ of a path $\vec{x}$, thinking of a limit of scaled secant vectors. Taking the limit of these vectors, our derivative gives us a vector which is tangent to the path.

PICTURE

The direction of $\vec{x}'$ gives us the direction of instantaneous of a particle moving along the path, and the length of $\vec{x}'$ tells us the speed of the particle. Recall that we sometimes refer to $\vec{x}'$ as the velocity vector, and write it as $\vec{v}$.

\begin{definition}
Consider a path $\vec{x}:I\subset\mathbb{R}\rightarrow\mathbb{R}^n$.

The \emph{velocity vector} of $\vec{x}$ at $t$ is $\vec{v}(t) = \vec{x}'(t)$. The velocity vector is tangent to $\vec{x}$ at $\vec{x}(t)$.

The \emph{speed} of $\vec{x}$ at $t$ is $\|\vec{x}'(t)\| = \|\vec{v}(t)\|$.
\end{definition}

\begin{example}
Consider the path $\vec{x}(t) = (\cos(t), \sin(t))$ for $0\leq t \leq 2\pi$, which parametrizes the unit circle in $\mathbb{R}^2$. We previously computed the velocity of this path as
\[
\vec{v}(t) = \vec{x}'(t) = (-\sin(t), \cos(t)).
\]
We can then compute the speed of $\vec{x}$ as
\begin{align*}
\|\vec{x}'(t)\| &= \|(-\sin(t), \cos(t))\|,\\
&= \sqrt{(-\sin(t))^2 + (\cos(t))^2},\\
& = \sqrt{1},\\
& = 1.
\end{align*}

Consider the path $\vec{y}(t) = (\cos(t^2), \sin(t^2))$ for $0\leq t\leq \sqrt{2\pi}$. This also parametrizes the unit circle in $\mathbb{R}^2$. The velocity vector of this path is
\[
\vec{y}'(t) = \answer{(-2t\sin(t^2), 2t\cos(t^2))}.
\]
The speed of this path is
\[
\|\vec{y}'(t)\| = \answer{2t}.
\]

Although both of these paths parametrize the unit circle counterclockwise and starting and ending at $(1,0)$, they do so in different ways. The first path, $\vec{x}$, traverses the unit circle at constant speed. The second path, $\vec{y}$, travels very slowly at first, then the speed increases as it travels around the circle.
\end{example}

Consider a path $\vec{x}:I\subset\mathbb{R}\rightarrow\mathbb{R}^n$. The velocity of this path gives us a vector $\vec{x}'(t)$ tangent to the curve at $\vec{x}(t)$. The tangent line to $\vec{x}$ at $\vec{x}(t)$ passes through the point $\vec{x}(t)$ and is parallel to the vector $\vec{x}'(t)$. This allows us to parametrize the tangent line, however we need to be very careful to distinguish between the parameter for the \emph{line} and the parameter for the \emph{path}. We do this by taking the parameter for our curve to be $t_0$ at our chosen point, so we are working with the point $\vec{x}(t_0)$ and the tangent vector $\vec{x}'(t_0)$.

\begin{proposition}
Consider a path $\vec{x}:I\subset\mathbb{R}\rightarrow\mathbb{R}^n$. We can parametrize the line tangent to $\vec{x}$ at $\vec{x}(t_0)$ as
\[
l(t) = \vec{x}(t_0) + t\vec{x}'(t_0) \textrm{ for } -\infty < t < \infty.
\]
\end{proposition}

Note that it's particularly important to allow the parameter $t$ to be any real number, otherwise we will be missing part of the line.

\section{Conclusion}

In this activity, we learned how to find the velocity and speed of a parametrized curve in $\mathbb{R}^n$.

\end{document}