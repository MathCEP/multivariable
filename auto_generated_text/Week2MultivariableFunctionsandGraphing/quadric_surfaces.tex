\documentclass{ximera}
\title{Graphing Functions}
\begin{document}
\begin{abstract}
\end{abstract}
\maketitle


In this activity, we introduce and classify quadric surfaces, which form an important family of surfaces.

\section{Definition of a Quadric Surface}

You might remember studying conic sections, such as parabolas, circles, ellipses, and hyperbolas. These are curves in the plane that arise through polynomial equations of degree two in two variables.

PICTURE

Quadric Surfaces are the three dimensional analogue of conic sections. That is, a quadric surface is the set of points in $\mathbb{R}^3$ satisfying some polynomial of degree two in three variables.

\begin{definition}
A \emph{quadric surface} is the set of points $(x,y,z)$ in $\mathbb{R}^3$ satisfying the equation
\[
Ax^2 + Bxy + Cxz + Dy^2 + Eyz + Fz^2 +Gx + Hy + Iz + J = 0,
\]
where $A,B,C,D,E,F,G,H,I,J\in\mathbb{R}$ are constants.
\end{definition}

Dealing with quadric surfaces in general can be computationally cumbersome, so we'll focus on quadric surfaces in some simple forms.

\begin{example}
The set of points satisfying
\[
x^2/a^2 + y^2/b^2 + z^2/c^2 = 1,
\]
for some constants $a,b,c\in\mathbb{R}$, is called an \emph{ellipsoid}.

PICTURE

An ellipsoid is kind of like a three dimensional ellipse. In fact, the sections and contour curves of such an an ellipsoid are ellipses.

In the special case that $a=b=c$, this ellipsoid is a sphere of radius $a$.
\end{example}

\begin{example}
The set of points satisfying
\[
z/c = x^2/a^2 + y^2/b^2,
\]
for some constants $a,b,c\in\mathbb{R}$, is called an \emph{elliptic paraboloid}.

PICTURE

The contour curves of such an elliptic paraboloid are ellipses, however the sections are parabolas which all open in the same direction.
\end{example}

\begin{example}
The set of points satisfying
\[
z/c = y^2/b^2 - x^2/a^2,
\]
for some constants $a,b,c\in\mathbb{R}$, is called a \emph{hyperbolic paraboloid}.

PICTURE

The contour curves of such a hyperbolic paraboloid are hyperbolas, and the sections are parabolas opening in opposite directions for $x$ and $y$ sections. This surface is often described as a ``saddle''.
\end{example}

\begin{example}
The set of points satisfying
\[
z^2/c^2 = x^2/a^2 + y^2/b^2,
\]
for some constants $a,b,c\in\mathbb{R}$, is called an \emph{elliptic cone}.

PICTURE

The contour curves of such an elliptic cone are ellipses, and the sections by $x=0$ and $y=0$ are pairs of intersecting lines.
\end{example}

\begin{example}
The set of points satisfying
\[
x^2/a^2 + y^2/b^2 - z^2/c^2 =1.
\]
for some constants $a,b,c\in\mathbb{R}$, is called a \emph{hyperboloid of one sheet}.

PICTURE

The contour curves of such a hyperboloid are ellipses, and the sections are hyperbolas.
\end{example}

\begin{example}
The set of points satisfying
\[
z^2/c^2 -x^2/a^2 - y^2/b^2 =1.
\]
for some constants $a,b,c\in\mathbb{R}$, is called a \emph{hyperboloid of two sheets}.

PICTURE

The contour curves of such a hyperboloid are ellipses, and the sections are hyperbolas. We describe this as the hyperboloid ``of two sheets'' since it has two disconnected pieces, as opposed to the hyperboloid of one sheet, which has only one.
\end{example}

APPLET/INTERACTIVE VARYING PARAMETERS

\section{Conclusion}

In this activity, we introduced and classified quadric surfaces, which form an important family of surfaces.

\end{document}